\documentclass[11pt]{article}
\usepackage[utf8]{inputenc}	% Para caracteres en español
\usepackage{amsmath,amsthm,amsfonts,amssymb,amscd}
\usepackage{multirow,booktabs}
\usepackage[table]{xcolor}
\usepackage{fullpage}
\usepackage{lastpage}
\usepackage{enumitem}
\usepackage{fancyhdr}
\usepackage{mathrsfs}
\usepackage{wrapfig}
\usepackage{setspace}
\usepackage{calc}
\usepackage{multicol}
\usepackage{cancel}
\usepackage[retainorgcmds]{IEEEtrantools}
\usepackage[margin=3cm]{geometry}
\usepackage{amsmath}
\newlength{\tabcont}
\setlength{\parindent}{0.0in}
\setlength{\parskip}{0.05in}
\usepackage{empheq}
\usepackage{framed}
\usepackage[most]{tcolorbox}
\usepackage{xcolor}
\colorlet{shadecolor}{orange!15}
\parindent 0in
\parskip 12pt
\geometry{margin=1in, headsep=0.25in}
\theoremstyle{definition}
\newtheorem{defn}{Definition}
\newtheorem{reg}{Rule}
\newtheorem{exer}{Exercise}
\newtheorem{note}{Note}
\newtheorem{prop}{Proposition}
\newtheorem{ex}{Example}
\newcommand{\brar}{$[-R,R]$}
\newcommand{\R}{\mathbb{R}}                      % Wes's shortcut command for the set of all real numbers
\newcommand{\C}{\mathbb{C}}                      % Wes's shortcut command for the set of all complex numbers
\usepackage{enumerate}
\renewcommand\thesection{\S\arabic{section}}
\newtheorem{theorem}{Theorem}
\newtheorem{lem}{Lemma}
\begin{document}


\thispagestyle{empty}

\begin{center}
{\LARGE \bf Chapter 3: Compact and Connected Sets}\\

\end{center}
\section{Compact Sets}
\subsection{Preliminary Ideas}
Some definitions before we give a precise definition for a compact set in $\R^{n}$.
\defn A set $A\subset \R^n$ is \textit{bounded} if and only if there is a constant $M\geq 0$ such that $A\subset D(0,M)$.

You can think of this as a set $A$ is bounded (that is, does not go one indefinitely) if and only if we can find some circle around the origin large enough to contain all of $A$. Another way to think of this is that for all $x\in A$, we can find an $M$ such that $\|x\|\leq M$.

\defn A \textit{cover} of a set $A$ is a collection $\{U_i\}$ of sets whose union contains $A$. If each set $U_i$ in the cover $\{U_i\}$ is open, then we say the collection $\{U_i\}$ is an \textit{open cover} of $A$. A \textit{subcover} of a given cover is just a subcollection of sets whose union also contains $A$, or as we say, \textit{covers} A. A subcover is a \textit{finite subcover} if the subcollection contains only a finite number of sets.

\ex The set of discs $\{D((x,0), 1) | x\in\R\}$ in $\R^2$ covers the real axis, and the subcollection of all discs $D((n,0),1)$ centered an integer points on the real line form a subcover. However, the same subcollection centered only at even integer points on the real line does not form a subcovering. this is becuase the discs are open sets so they area their boundaries touch do not contain what we want them to (e.g. (1,0)).

\note Open covers are not necessarily countable collections of open sets.

We now state the main theorem with an associated definition!

\subsection{The Heine-Borel and Bolzano-Weirstrass Theorems}
\begin{shaded}
\theorem Let $A\subset \R^n$. Then the following conditions are equivalent:
    \begin{enumerate}[(i)]
        \item A is closed and bounded.
        \item Every open cover of $A$ has a finite subcover.
        \item Every sequence in $A$ has a subsequence which converges to a point of $A$.
    \end{enumerate}
The equivalence of (i) and (ii) constitute the Heine-Borel theorem while the equivalence of (i) and (iii) form the Bolzano-Weirstrass theorem.
\end{shaded}
\note For metric spaces, in general (ii) and (iii) are equivalent but (i) is \textit{not} equivalent to (ii) and (iii). The equivalenve is a \textit{special} and very important property of $\R^n$.

Notice that the Bolzano-Weirstrass theorem is reasonably intuitive: if $A$ is bounded, then any subsequence of points must bunch up somewhere, and if $A$ is closed, that somewhere must be somewhere in $A$.

The Heine-Borel theorem is less intuitive. Let's take a look at some examples.
\ex The entire real line is not compact becase it is closed but not bounded. In terms of Heine-Borel, this means that there exists some open cover of $A$ such that it has no finite subcover of $A$. Consider that 
$$
\{D(n,1)\; | \;n\in \mathbb{Z}\}
$$
forms an open cover for $A$ but it has no finite subcover.

\ex Consider the set $A=(0,1]$. Evidently, it is not compact because it is not closed. Thus, we should be able to find some cover of $A$ such that it does not have a finite subcover. Consider the open cover
$$
\{(1/n,2)\;|\; n\in\mathbb{N}\}.
$$
Notice this cover has no finite subcover for $A$.

There are two other ways of formulating (ii) and (iii):
\begin{enumerate}[(i)']
\addtocounter{enumi}{1}
    \item Every collection of closed sets with the finite intersection property for $A$ has a non-empty intersection with $A$.
    \item Every infinite subset of $A$ has an accumulation point in $A$.
\end{enumerate}

Let us now prove theorem 1. We shall prove that (i) $\Rightarrow$ (ii) $\Rightarrow$ (iii) $\Rightarrow$ (i). First, we show (i) $\Rightarrow$ (ii). We begin by proving a special case.
    \lem The Heine-Borel property (ii) holds for closed intervals $[a,b]$ in $\R$.
\begin{proof}
    Let $\mathcal{U}=\{U_i\}$ be an open covering of $[a,b]$ and define the set $A=\{x\in[a,b]\;|\; \mbox{the set $[a,x]$}\\
    \mbox{can be covered by a finite collection of the } U_i\}$. In other words, $A$ is the set of all intervals in $[a,b]$ starting at $a$ such that $U_i$ has a finite subcover for said interval. Then we want to show that $A=[a,b]$, since this would be the same as saying any open cover of $[a,b]$ has a finite subcover. To this end, let $c=\sup (A)$. The sup exists because $A\neq\emptyset$ (as $a\in A$) and $A$ is bounded above by $b$. Also, since $[a,b]$ is closed, $c\in [a,b]$ (since a closed set includes its boundaries). Now, suppose $c\in U_{i_0}$; such a $U_{i_0}$ exists since $[a,b]\subset\{U_i\}$. Since $U_{i_0}$ is open, here exists an $\varepsilon > 0$ such that $(c-\varepsilon,c+\varepsilon)\subset U_{i_0}$. Since $c=\sup(A)$, there exists an $x\in A$ such that $c-\varepsilon < x\leq c$. Because $x\in A$, $[a,x]$ has a finite subcover, say $U_1,\dots,U_N$; then $[a,c+\varepsilon/2]$ also has the finite subcover $U_1,\dots,U_N,U_{i_0}$. Thus, we conclude that $c\in A$ and moreover, $c=b$. Indeed, if $c<b$, we would get a member of $A$ larger than c, since $[a,c+\varepsilon/2]$ has a finite subcover. The latter cannot happen since $c=\sup(A)$.
\end{proof}
    \lem In $A\subset \R^n$ is compact and $x_0\in \R^m$ then $A\times \{x_0\}$ is compact.
\begin{proof}
    Let $\mathcal{U}$ be an open cover of $A\times\{x_0\}$, and $\mathcal{V}=\{V\;|\; V= \{y\;|\; (y,x_0)\in U\}, U\in\mathcal{U}\}$. Then $\matchcal{V}$ is an open cover of $A$ in $\R^n$, and hence, $\mathcal{V}$ has a finite subcover of $A$, $\mathcal{V}^'=\{V_1,\dots,V_k\}$. Each $V_i\in\mathcal{V}^'$ corresponds to a $U_i\in \mathcal{U}$, and $\mathcal{U}=\{U_1,\dots,U_k\}$ is then a finite subcover in $\mathcal{U}$ of $A\times\{x_0\}$.
\end{proof}
That proof is a little hard to understanding but the basic idea is that because $A$ is closed, we can form some open finite subcover for $A$ and just extend each set in the subcover to contain the point $x_0$ in the subcover, creating an open finite subcover for $A\times\{x_0\}$. Moving on.
\lem If $[-R,R]^{n-1}\subset \R^{n-1}$ has the Hiene-Borel property, then $[-R,R]^n\subset \R^n$ has the Heine-Borel property, where $[-R,R]^n=[-R,R]\times\cdots\times[-R,R]$, $n$ times.

\begin{proof}
    Suppose $[-R,R]^{n-1}$ has the Heine-Borle property and that $\mathcal{U}$ is an open cover of $\brar^n$. Let $S=\{x\in\brar\;|\;\brar^{n-1} \times[-R,x] \mbox{ can be covered by a finite number of set in $\mathcal{U}$} \} $. Now, $-R\in S$, since $\brar^{n-1}$ satisfies (ii) by hypothesis, and so by Lemma 2 we know $\brar^{n-1}\times {-R}$ has a finite subcover in $\mathcal{U}$. Also, $S$ is bounded above by $R$ and therefore $S$ has a supremum, say $x_0$. We will show that $x_0=R$, which will prove the lemma.

    Let $\mathcal{U}^{'} \subset\mathcal{U}$ be a finite subcover of $\brar^{n-1}\times\{x_0\}$. For each $(y,x_0)\in[-R,R]^{n-1}\times\{x_0\}$ there exists $\varepsilon_y>0$ such that $D((y,x_0),\sqrt2 \varepsilon_y)$ is coverd by $\mathcal{U}^'$. But $V_y=D(y,\varepsilon_y) \times (x_0-\varepsilon_y,x_0+\varepsilon_y)\subset D((y,x_0),\sqrt{2}\varepsilon_y)$ so $V_y$ is covered by $\mathcal{U}^'$. Consider the open cover $\mathcal{V}=\{V_y\;|\; y\in\brar^{n-1}\}$ of $\brar^{n-1}\times \{x_0\}$. By Lemma 2, $\mathcal{V}$ has a finite subcover of $\brar^{n-1}\times \{x_0\}$, say $\{V_{y1},\dots,V_{yN}\}$. Let $\varepsilon=\inf\{\varepsilon_{y1},\dots,\varepsilon_{yN}\}$. Then $\brar^{n-1}\times(x_0-\varepsilon,x_0+\varepsilon)\subset \cup_{i=1}^N V_{yi} $, and so $\brar^{n-1}\times(x_0-\varepsilon,x_0+\varepsilon)$ is covered by $\mathcal{U}^'$. 
    
    Now, with this $\varepsilon$, there exists some $x\in S$ such that $x_0-\varepsilon<x\leq x_0$. Since $x\in S$, there exists a finite subcover $\mathcal{U}^{''}\subset \mathcal{U}$ which covers $\brar^{n-1}\times [-R,x]$, and $\mathcal{U}^{'}\cup \mathcal{U}^{''}$ is a finite cover of $\brar^{n-1}\times [-R,x_0+\varepsilon)$. Thus, $x_0\in S$. Suppose $x_0<R$, then choose $\delta$ such that $x+\delta<R$ and $x_0+\delta<x_0+\epsilon$. Thus, $\brar^{n-1}\times [-R,x_0+\delta]$ is covered by $\mathcal{U}^{''} \cup \mathcal{U}^'$, and $x_0+\delta\in S$, a contradiction, and therefore $X_0\in R$. 
\end{proof}
Just one more lemma, hang in there.
\lem If $A$ satisfies (ii), $B$ is closed and $B\subset A$, then $B$ also satisfies (ii).
\begin{proof}
    Let $\{U_i\}$ be an open covering of $B$, and let $V=\R^n \symbol{92} B$. Then $\{U_i,V\}$ is an open cover of $A$. If $\{U_1,\dots,U_N,V\}$ is a finite subcover of $A$, then $\{U_1,\dots,U_N\}$ is a finite subcover for $B$.
\end{proof}
Finally, we prove Heine-Borel. If (i) then (ii).
\begin{proof}
    If $A$ bounded, then it lies in some cube $\brar^n$. By Lemma 3 and induction on $n$, this cube satisfies (ii). By Lemma 4, $A$ does also, since $A$ is closed.
\end{proof}
That was real underwhelming, wasn't it. Now to prove that (ii) implies (iii).
\begin{proof}
    Suppose for the sake of contradiction, suppose that the sequence $x_k\in A$ has \textit{no} convergent subsequences. In particular, this means that $x_k$ has an infinity of distinct points, say, $y_1,y_2,\dots$. Since there are no convergent subsequences, there is a neighborhood $U_k$ of $y_k$ containing no other $y_j$. This is because if every neighborhood of $y_k$ contained another $y_j$, we could, by choosing the neighborhoods $D(y_k,1/m), m=1,2,\dots$ select out a subsequence converging to $y_k$. Also, we claim that the set $y_1,y_2,\dots$ is closed. Indeed, it has no accumulation points by the assumption that there are no convergent subsequences. Now, by Lemma 3 above, $\{y_1,y_2,\dots\}$ satisfies (ii). But $\{U_k\}$ is an open cover which has no finite subcover, a contradiction. Thus, $x_k$ must has a convergent subsequence. Now we only have to show that the limit lies within $A$ to show that it is closed. Figure that yourself loser!
\end{proof}
Finally, we prove that (iii) implies (i).
\begin{proof}
    First, we show that $A$ is closed. For this, we use the theorem stating that a set $A\subset \R^n$ is closed if and only if for every sequence $x_k\in A$ which converges, the limit lies in $A$. Consider the sequence $x_k\to x$ with $x_k\in A$. By (iii), the limit lies in $A$, so $A$ is closed.

    Now, we prove $A$ is bounded. Suppose the contrapositive, that $A$ is not bounded. Then there are points $x_k\in A$ with $\|x_k\|\geq k,k=1,2,\dots$. This implies that the sequence $x_k$ cannot have any convergent subsequences since, if $y$ was a limit point, $\|y\|=\lim_{k\to\infty}\|x_k\|=\infty$. This is impossible if we have $y\in\R^n$.
\end{proof}
\section{Nested Set Property}
An important consequence of Theorem 1 is called the \textit{nested set property}.
\theorem Let $F_k$ be a sequence of compact non-empty sets in $\R^n$ such that $F_{k+1}\subset F_k$ for all $k=1,2,\dots$. Then there is at least one point in $\bigcap_{k=1}^\infty F_k$.
\begin{proof}
    Let us observe that in the compact set $A=F_1$, the sets $F_1,F_2,\dots$ have the finite intersection property. Indeed, the intersection of any finite collection equals the $F_k$ with the highest index. Thus, since (ii)' holds for compact sets, we have 
    $$
    F_1\cap \left(\bigcap_{k=1}^\infty F_k\right)=\bigcap_{k=1}^\infty\{F_k\}\neq\emptyset.
    $$
\end{proof}
    Intuitively, this makes sense as the sets are getting smaller and smaller but still contain something. However, if the $F_k$ are non-compact, you can imagine that this proof would require a little more than just that.
\note Theorem 2 is not true if "compact non-empty" is replaced by "open non-empty". Indeed, consider the sequence of open non-empty sets $F_k=(k,\infty)$ or $[k,\infty)$.
    \ex Verify theorem 2 for $F_k=[0,1/k]\subset \R$.

    \textit{Solution:} Each $F_k$ is compact and clearly, $F_{k+1}\subset F_k$. The intersection is $\{0\}$ which is non-empty.
\section{Path-Connected Sets}
Intuitively, we have a decent sense of what "connectedness" means for a set. No gaps. But this intuition can get hazy with more complicated sets, for instance, is the set $\{(x,\sin 1/x)\;|\;x>0\}\cup\{(0,y)\;|\;y\in[1,0]\}$ in $\R^2$ connected? Therefore, we want a more formal definition.

There are two different but closely-related notions of connectedness. The more intuitive and applicable of these is that of path-connectedness, so we begin there. Our definition first define what is meant by a curve (path) joining two points.
\defn A \textit{continuous path} joining two points $x$ and $y$ in $\R^n$ is a mapping $\varphi:[a,b]\to\R^n$ such that $\varphi(a)=x, \varphi(b)=y$, and $\varphi$ is continuous. Here, $x$ may or may not equal $y$ and $b\geq a$.
\note For now, we will use a rough definition of continuous: a mapping $\varphi$ is continuous if 
$$
(t_k\to t)\Rightarrow (\varphi(t_k)\to\varphi(t))
$$
for every sequence $t_k$ in $[a,b[$ converging to some $t\in[a,b]$. Also, a path $\varphi$ is said to \textit{lie in a set} $A$ if $\varphi(t)\in A$ for all $t\in[a,b]$.

Given this, we say that a set $A$ is \textit{path-connected} if any two points in the set can be joined by a continuous path lying in the set $A$. The set we described above is therefore not path-connected (though the proof is somewhat difficult).

\ex The set $[0,1]$ is path-connected.
\begin{proof}
    For any two points $x,y\in[0,1]$, define the mapping $\varphi(t)=(y-x)t+x$. This is a path connected $x$ and $y$, and it lies in $[0,1]$.
\end{proof}
\ex Must a path-connected set be closed? Or open?

\textit{Solution:} No; $[0,1]$, $[0,1)$, and $(0,1)$ are all path-connected sets.

\ex Let $\varphi:[0,1]\to \R^3$ be a continuous path, and $\mathcal{C}=\varphi([0,1])$. Show that $\mathcal{C}$ is path-connected.

\textit{Solution:} This is intuitively clear, for we can use the path $\varphi$ itself to join any two points on $\mathcal{C}$. Precisely, if $x=\varphi(a)$ and $y=\varphi(b)$, where $0\leq a\leq b \leq 1$, let $c:[a,b]\to \R^3$, $c(t)=\varphi(t)$. Then, $c$ is a path joining $x$ to $y$ and $c$ lies in $\mathcal{C}$.

\section{Connected Sets}
We can now define what it means for a set to be connected.
\defn A set $A\subset \R^n$ is called \textit{connected} if there do not exists two non-empty, open set $U, V$ such that:
\[A\subset U\cup V\]
\[A\cap U\neq \emptyset\]
\[A\cap V\neq \emptyset\]
\[A\cap U\cap V= \emptyset\].

Intuitively, this definition is stating that if $A$ is connected, we can't find two set $U$ and $V$ that would separate $A$ into two pieces. Notice then that the set at the beginning of \S3 is connected but not path-connected. Then these two notions are not the same. However, there is thie following relation:
\theorem If a set $A$ is path-connected, then $A$ is connected.

Let us first start with a special case:
\lem The interval $[a,b]$ is connected.
\begin{proof}
    Suppose, for the sake of contradiction, that the interval were not connected. Then there would be two open set $U$ and $V$ with $U\cap [a,b]\neq \emptyset$, $V\cap [a,b]\neq\emptyset$, $[a,b]\cap U\cap V = \emptyset$, and $[a,b]\subset U\cup V$. Further, suppose that $b\in V$. Let $c=\sup(U\cap [a,b])$, which exists as the set is bounded above. Now $U\cap [a,b]$ is closed since its complement is $V\cup (\R\symbol{92}[a,b])$, which is open. Thus, $c\in U\cap [a,b]$. Now, $c\neq b$, since $c\not\in V$ and $b\in V$. Any neighborhood of $c$ intersects $V\cap [a,b]$ since $c\neq b$ and no neighborhood of $c$ can be entirely contained in $U$ as $c=\sup(U\int [a,b])$, so that $c$ is an accumulation point of $V\cap[a,b]$. But as with $U,V\cap[a,b]$ is closed, so $c\in V\cap [a,b]$. This contradicts the fact that $V\cap U\cap [a,b]\neq \emptyset$.
\end{proof}

Now we prove Theorem 3.
\begin{proof}
    Suppose that $A$ is path-connected and, for the sake of contradiction, that $A$ not connected. Then, by definition, there exists open sets $U,V$ such that $A\subset U\cup V$, $A\cap U \cap V= \emptyset$, $A\cap U \neq \emptyset$, and $A\cap V\neq\emptyset$. Choose some $x\in U\cap A$ and $y\in V\cap A$. Since $A$ is path-connected, there exists a path $\varphi:[a,b]\to\R^n$ in $A$ joining $x$ and $y$. Set $U_0=\varphi^{-1}(U)$ and $V_0=\varphi^{-1}(V)$ so $U_0,V_0\subset[a,b]$. Now $U_0$ is closed, because if we let $t_k\to t$, with $t_k\in U_0$, then, by the continuity of $\varphi$, $\varphi(t_k)\to \varphi(t)$; but since $V$ is open, $\varphi(t)\not\in V$, or else $\varphi(t_k)\in V$ for large $k$. Hence $\varphi(t)\in U\cap A$ or $t\in U_0$. Thus $U_0$ is closed. Similarly, $V_0$ is closed. Let $U^{'}=(-\infty,a)\cup (\R\symbol{92} V_0$), and $V^{'}=(b,\infty)\cup (\R\symbol{92} U_0)$, which are open sets. Observe that $U^{'}\cap [a,b]\neq\emptyset$, $V^{'}\cap[a,b]\neq\emptyset$, $U^{'}\cap V^{'}=\emptyset$, and $[a,b]\subset U^'\cup V^'$. Thus, $[a,b]$ is not connected, contradicting Lemma 5.
\end{proof}
This is usually the easiest way to identify a connected set. Finally, if a set $A$ is not connected (and hence not path-connected), we can divide it into piece, called components. More precisely, a \textit{component} of a set $A$ is a connected subset $A_0\subset A$ such that there is no connected set in $A$ containing $A_0$ other than $A_0$ itself. 

\end{document}