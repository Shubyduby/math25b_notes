\documentclass[11pt]{article}
\usepackage[utf8]{inputenc}	% Para caracteres en español
\usepackage{amsmath,amsthm,amsfonts,amssymb,amscd}
\usepackage{multirow,booktabs}
\usepackage[table]{xcolor}
\usepackage{fullpage}
\usepackage{lastpage}
\usepackage{enumitem}
\usepackage{fancyhdr}
\usepackage{mathrsfs}
\usepackage{wrapfig}
\usepackage{setspace}
\usepackage{calc}
\usepackage{stmaryrd}
\usepackage{multicol}
\usepackage{cancel}
\usepackage[retainorgcmds]{IEEEtrantools}
\usepackage[margin=3cm]{geometry}
\usepackage{amsmath}
\newlength{\tabcont}
\setlength{\parindent}{0.0in}
\setlength{\parskip}{0.05in}
\usepackage{empheq}
\usepackage{framed}
\usepackage[most]{tcolorbox}
\usepackage{xcolor}
\colorlet{shadecolor}{orange!15}
\parindent 0in
\parskip 12pt
\geometry{margin=1in, headsep=0.25in}
\theoremstyle{definition}
\newtheorem{defn}{Definition}
\newtheorem{reg}{Rule}
\newtheorem{exer}{Exercise}
\newtheorem{note}{Note}
\newtheorem{prop}{Proposition}
\newtheorem{ex}{Example}
\newcommand{\R}{\mathbb{R}}                      % Wes's shortcut command for the set of all real numbers
\newcommand{\C}{\mathbb{C}}                      % Wes's shortcut command for the set of all complex numbers
\newcommand{\Q}{\mathbb{Q}}
\newcommand{\N}{\mathbb{N}}
\usepackage{enumerate}
\renewcommand\thesection{\S\arabic{section}}
\newtheorem{theorem}{Theorem}
\newtheorem{lem}{Lemma}
\usepackage{mdframed}
\newcommand{\bslash}{\symbol{92}}
\newcommand{\upint}[1][2]{\overline{\int_{#1}^{#2}}}
\newcommand{\loint}[1][2]{\underline{\int_{#1}}^{#2}}
\usepackage{dirtytalk}
\newcommand{\bconditions}{\left\{\begin{aligned}}
\newcommand{\econditions}{\end{aligned}\right.}
\newcommand{\mat}{\begin{bmatrix}}
\newcommand{\trix}{\end{bmatrix}}

\newcommand{\dell}{\partial}
\begin{document}


\thispagestyle{empty}

\begin{center}
{\LARGE \bf Lecture 36: Fourier Series, Bessel's Inequality, Parseval's Theorem, Mean Completeness}\\
{\large Monday, 24  April 2023}\\
\end{center}
\section{Fourier Series}
Recall that given a finite dimension inner product space and an orthonormal basis $u_1,\dots,u_n$, for any $v\in V$ we have that
$$
v=\sum_{k=1}^n \langle v,u_k\rangle u_k.
$$
\begin{mdframed}[backgroundcolor = blue!10]
\vspace{+0.1cm}
\defn If $\phi_0,\phi_1,\phi_2,\dots$ is an orthonormal sequence in a real inner product space, V and $f\in V$, we write
$$
f=\sum_{k=0}^\infty c_k\phi_k 
$$
to mean
$$
\lim_{n\to \infty} \left\| f -\sum_{k=0}^n c_k\phi_k \right\| =0,
$$
where $\|\cdot\|$ is the norm induced by the inner product.

\end{mdframed}
\prop Suppose $f\in V$ and
$$
f=\sum_{k=0}^\infty  c_k\phi_k
$$
for some orthonormal sequence $\phi_0,\phi_1,\phi_2,\dots$. Then $c_k = \langle f, \phi_k\rangle$ for all $k$.

\begin{proof}
    Let $s_n$ denote the $n$th partial sum of the summation above and some $\varepsilon>0$ be given. Then there exists an $N\in\N$ such that $\|f-s_n\|<\varepsilon$ for all $n\geq N$. Given any $k\in \{0,1,2,\dots\}$, pick $n\geq \max \{k,N\}$ and consider
    $$
    \langle f,\phi_k\rangle = \langle f-s_n+s_n,\phi_k\rangle = \langle f-s_n,\phi_k\rangle +\underbrace{\langle s_n,\phi_k\rangle}_{c_k},
    $$
    so $c_k=\langle s_n,\phi_k\rangle$ because $s_n$ is a finite sum, $n\geq k$, and $\phi_0,\dots,\phi_n$ are orthonormal. Thus,
    $$
    |\langle f, \phi_k\rangle  -c_k| = |\langle f-s_n,\phi_k\rangle|\leq \|f-s_n\|\|\phi_k\|<\varepsilon.
    $$
\end{proof}
\begin{mdframed}[backgroundcolor = blue!10]
\vspace{+0.1cm}
\defn Suppose $\phi_0,\phi_1,\phi_3,\dots$ is an orthonormal sequence in the real inner product space $V$. This sequence is a \textit{complete} orthonormal sequence if every $f\in V$ can be written in the form 
$$
f=\sum_{k=0}^\infty c_k\phi_k.
$$

\defn The series above,
$$
\sum_{k=0}^\infty \langle f,\phi_k\rangle \phi_k
$$
is the \textit{Fourier series} for $f$.
\end{mdframed}
\ex In $L^2[-\pi,\pi]$, recall that the sequence
$$
\phi_0 = \frac{1}{\sqrt{2\pi}},\quad \phi_1=\frac{\sin(x)}{\sqrt{\pi}},\quad \phi_2 = \frac{\cos(x)}{\sqrt{\pi}},\quad\phi_3=\frac{\sin(2x)}{\sqrt{\pi}},\quad \phi_4=\frac{\cos(2x)}{\sqrt{2\pi}},\dots
$$
form an orthonormal sequence. Consider the functions
$$
f(x) = -2+7\cos(2x),\qquad g(x) =\begin{cases}
f(x), &x\in(\R\bslash \Q)\cap [-\pi,\pi]\\
0, &x\in\Q\cap[-\pi,\pi].
\end{cases}
$$
Notice that for both $f$ and $g$,
$$
c_0=-2\sqrt{2\pi},\quad c_4 = 7\sqrt{\pi}.
$$
The funcitons $f$ and $g$ differ on a set of measure 0, namely $\Q\cap[-\pi,\pi]$. So $f\sim g$.

\section{Bessel's Inequality, Parseval's Theorem}

\prop (\textit{Bessel's Inequality}) If $\phi_0,\phi_1,\phi_2,\dots$ is an orthonormal sequence in a real inner product space $V$ then for each $f\in V$, the sum
$$
\sum_{k=0}^\infty |\langle f,\phi_k\rangle |^2
$$
converges and is at most $\|f\|^2$.

\begin{proof}
    Let 
    $$
    s_n=\sum_{k=0}^n \langle f,\phi_k\rangle \phi_k
    $$
    denote the partial sums of the series. We first want to show that $f-s_n$ is orthogonal to $s_n$. If we choose some $j\in\{1,2,\dots, n\}$, the $j$th term in $s_n$ is $\langle f,\phi_j\rangle \phi_j$. Then 
    $$
    \begin{aligned}
    \langle f-s_n,\langle f,\phi_j\rangle \phi_j\rangle&=\langle f,\phi_j\rangle\langle f-s_n,\phi_j\rangle =\langle f,\phi_j\rangle^2  -\langle f,\phi_j\rangle\langle s_n, \phi_j\rangle.
    \end{aligned}
    $$
    Notice that
    $$
    \left\langle \sum_{k=0}^n \langle f,\phi_k\rangle \phi_k,\phi_j \right\rangle=\langle f,\phi_j\rangle.
    $$
    which implies that $\langle f-s_n, s_n\rangle =0$.
    By the Pythagorean theorem, we have 
    $$
    \|f\|^2=\|f-s_n\|^2+\|s_n\|^2,
    $$
    so for each $n$,
    $$
    \|s_n\|^2\leq \|f\|^2.
    $$
    Note that
    $$
    \|s_n\|^2 = \sum_{k=0}^n\langle f,\phi_k\rangle^2,
    $$
    so the sequence of partial sums $\left\{\|s_n\|^2\right\}_{n=0}^\infty$ is monotone increasing and bounded above by $\|f\|^2$. Then by the monotone convergence theorem, the infinite sum converges and
    $$
    \sum_{k=0}^\infty \langle f,\phi_k\rangle^2 \leq\|f\|^2.
    $$
\end{proof}
\note We did not assume $\phi_0,\phi_1,\dots$ was complete. 
\prop (\textit{Parseval}) Let $\phi_0,\phi_1,\phi_2,\dots$ be an orthonormal sequence in an inner product space $V$. Then the orthonormal sequence is complete if and only if for each $f\in V$, we have equality in Bessel's inequality rather than inequality.

\begin{proof}
    If the orthonormal sequence in the assumption is complete, we have that
    $$
    \|f\|^2 = \|f-s_n\|^2 +\|s_n\|^2
    $$
    from the proof of Bessel's inequality. As $n\to \infty$, $\|f-s_n\|\to 0$, so 
    $$\|f\|^2=\lim_{n\to \infty} \|s_n\|^2=\sum_{k=0}^\infty \langle f,\phi_k\rangle ^2.$$
\end{proof}

\begin{shaded}
\theorem Let $V$ be an inner product space and $\phi_0,\phi_1,\dots,\phi_n$ be a set of orthonomal vectors. Then for every list of real numbers $\alpha_0,\alpha_1,\dots,\alpha_n$,
$$
\left\| f - \sum_{k=0}^n \alpha_k\phi_k \right\|\geq \left\| f- \sum_{k=0}^n \langle f,\phi_k\rangle \phi_k\right\|
$$
with equality if and only if $\alpha_k=\langle f,\phi_k\rangle$.
\end{shaded}

\begin{proof}
    This a result of comparing the sequences of norms.
\end{proof}
\ex Let $f(x)=x-x^2$ on $[-\pi,\pi]$ and consider the orthonormal set 
$$
\phi_0 =\frac{1}{\sqrt{2\pi}},\quad \phi_1=\frac{\sin{x}}{\sqrt{\pi}},\quad \phi_2 = \frac{\cos{x}}{\sqrt{\pi}}.
$$
The best $L^2[-\pi,\pi]$ approximation of $f$ is
$$
\langle f,\phi_0\rangle \phi_0+\langle f,\phi_1\rangle \phi_1+\langle f,\phi_2\rangle\phi_2.
$$
Note that
$$
\langle f,\phi_0 \rangle = \int_{-\pi}^\pi (x-x^2)\frac{1}{\sqrt{2\pi}}~dx = \frac{\sqrt{2}}{3}\pi^{\frac{5}{2}},\quad \langle f,\phi_1\rangle = 2\sqrt{\pi},\quad \langle f,\phi_2\rangle = 4\sqrt{\pi}.
$$
Now graph the two functions are see how close they are!

\begin{shaded}
    \theorem (\textit{Mean completeness}) In the space $L^2[-\pi,\pi]$, the orthonormal sequence in Example 1 is complete.
\end{shaded}

\ex If we want to send a voice memo, we do not need to send the entire sound wave, just the Fourier coefficients for the sound wave!
\end{document}