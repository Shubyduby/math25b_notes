\documentclass[11pt]{article}
\usepackage[utf8]{inputenc}	% Para caracteres en español
\usepackage{amsmath,amsthm,amsfonts,amssymb,amscd}
\usepackage{multirow,booktabs}
\usepackage[table]{xcolor}
\usepackage{fullpage}
\usepackage{lastpage}
\usepackage{enumitem}
\usepackage{fancyhdr}
\usepackage{mathrsfs}
\usepackage{wrapfig}
\usepackage{setspace}
\usepackage{calc}
\usepackage{stmaryrd}
\usepackage{multicol}
\usepackage{cancel}
\usepackage[retainorgcmds]{IEEEtrantools}
\usepackage[margin=3cm]{geometry}
\usepackage{amsmath}
\newlength{\tabcont}
\setlength{\parindent}{0.0in}
\setlength{\parskip}{0.05in}
\usepackage{empheq}
\usepackage{framed}
\usepackage[most]{tcolorbox}
\usepackage{xcolor}
\colorlet{shadecolor}{orange!15}
\parindent 0in
\parskip 12pt
\geometry{margin=1in, headsep=0.25in}
\theoremstyle{definition}
\newtheorem{defn}{Definition}
\newtheorem{reg}{Rule}
\newtheorem{exer}{Exercise}
\newtheorem{note}{Note}
\newtheorem{prop}{Proposition}
\newtheorem{ex}{Example}
\newcommand{\R}{\mathbb{R}}                      % Wes's shortcut command for the set of all real numbers
\newcommand{\C}{\mathbb{C}}                      % Wes's shortcut command for the set of all complex numbers
\newcommand{\Q}{\mathbb{Q}}
\newcommand{\N}{\mathbb{N}}
\usepackage{enumerate}
\renewcommand\thesection{\S\arabic{section}}
\newtheorem{theorem}{Theorem}
\newtheorem{lem}{Lemma}
\usepackage{mdframed}
\newcommand{\bslash}{\symbol{92}}
\newcommand{\upint}[1][2]{\overline{\int_{#1}^{#2}}}
\newcommand{\loint}[1][2]{\underline{\int_{#1}}^{#2}}
\usepackage{dirtytalk}
\newcommand{\bconditions}{\left\{\begin{aligned}}
\newcommand{\econditions}{\end{aligned}\right.}
\newcommand{\mat}{\begin{bmatrix}}
\newcommand{\trix}{\end{bmatrix}}
\newcommand{\dell}{\partial}
\begin{document}


\thispagestyle{empty}

\begin{center}
{\LARGE \bf Lecture 31: Integration Techniques}\\
{\large Wednesday, 12 April 2023}\\
\end{center}

\section{Change of Variables from $\R\to\R$}
We did finishing proving this proposition from last class:
\prop (\textit{First substitution theorem}) Let $I=[a,b]$ and $\varphi:I\to \R$ be of class $C^1$. If $f$ is continuous on some interval $J$ containing $\varphi(I)$, then 
$$
\int_a^b f(\varphi(t))\varphi'(t)~dt=\int_{\varphi(a)}^{\varphi(b)} f(x)~dx.
$$

\begin{proof}
    If $c=\varphi(a)$ and $d=\varphi(b)$, then $J$ contains the interval with endpoints $c$ and $d$. Define $F:J\to \R$ by 
    $$
    F(u)=\int_c^u f(x)~dx
    $$
    and $g:I\to \R$ by $g(t)=F(\varphi(t))$. Then $g$ is differentiable and $g'(t)=F'(\varphi(t))\varphi'(t)=f(\varphi(t))\varphi'(t)$ so
    $$
    \begin{aligned}
    \int_a^b f(\varphi(t))\varphi'(t)~dt&=\int_a^b g'(t)~dt \\
    &=g(b)-g(a)\\
    &=F(\varphi(b))-F(\varphi(a))\\
    &=F(d)-F(c)\\
    &=\int_c^d f(x)~dx.
    \end{aligned}
    $$
\end{proof}
\prop (\textit{Second substitution theorem}) Let $I=[a,b]$ and $\varphi:I\to \R$ be of class $C^1$ and $\varphi'(t)\neq 0$ for all $(t)\in I$. If $f$ is continuous on $\varphi(I)$ then 
$$
\int_a^b f(\varphi(t))~dt=\int_{\varphi(a)}^{\varphi(b)} f(x)\left(\varphi^{-1}\right)'(x)~dx.
$$
\begin{proof}
    Note that since $\varphi$ is continuous on the compact, connected set $I$, then the set $J=\varphi(I)$ is also compact and connected. Moreover, the fact that $\varphi'$ is continuous and non-zero on $I$ means that $\varphi$ cannot change sign on $I$. Then $\varphi$ is strictly monotone on $I$ and $\varphi:I\to J$ is a bijection, where $J=\varphi(I)$. The inverse function theorem then guarantees that $\varphi^{-1}:J\to I$  is of call $C^1$ on $J$.

    Now, let $G:I\to \R$ be an antiderivative of $f\circ \varphi$. Then $G$ is differentiable on $I$ and, by the chain rule,
    $$
    (G\circ \varphi^{-1})'(x) = G'(\varphi^{-1}(x)) \cdot (\varphi^{-1})'(x) = f\circ \varphi(\varphi^{-1}(x))\cdot (\varphi^{-1})'(x)=f(x)(\varphi^{-1})'(x).
    $$
    By the fundamental theorem of calculus,
    $$
    \begin{aligned}
        \int_{\varphi(a)}^{\varphi(b)} f(x)\left(\varphi^{-1}\right)'(x)~dx &= (G\circ \varphi^{-1})'(\varphi(b)) -(G\circ \varphi^{-1})'(\varphi(b))\\
        &=G(b)-G(a)\\
        &=\int_a^b f(\varphi(t))~dt.
    \end{aligned}
    $$
\end{proof}

\ex Evaluate the following integral:
$$
\int_1^4 \frac{1}{1+\sqrt{t}}~dt.
$$

\textit{Solution:} Let $\varphi(t)=\sqrt{t}$. Then $\varphi^{-1}$ exists on this domain and $\varphi^{-1}(s)=s^2$. Let $t=s^2$. Then our new integral by Proposition 2 is
$$
\int_1^2 \frac{2s}{1+s}~ds=\int_1^2 2-\frac{2}{1+s}~ds.
$$
The rest should be easily solvable.

\section{Multivariate Change of Variables}

\begin{shaded}
\theorem Let $\varphi:U\to W$ be a $C^1$ diffeomorphism, where $U$ and $W$ are open subsets of $\R^n$. Let $R\subseteq U$ be a rectangle and $f:W\to R$ be bounded and integrable. Then then the composition $(f\circ\varphi)\cdot |\det(D\varphi)|$ is integrable on $R$ and 
$$
\int_R (f\circ \varphi)|\det(D\varphi)| =\int_{\varphi(R)}f
$$
\end{shaded}
\begin{proof}
    This is only a sketch. Think about slicing $R$ into subrectangles that are \textit{small}. We want to compare the contributions to the integral area over a small slice, so pick some subrectangle $S$ with $x_0\in S$. Then we have that
    $$
    \int_S (f\circ\varphi)|\det(D\varphi)|\approx f(\varphi(x_0))\cdot|\det(D\varphi(x_0))|\cdot v(S)
    $$
    Looking the other way around, we have that
    $$
    \int_{\varphi(S)}f\approx f(\varphi(x_0))\cdot v(\varphi(S))\approx f(\varphi(x_0))\cdot |\det(D\varphi(x_0))|\cdot v(S),
    $$
    since at such a small scale every thing looks roughly linear.
\end{proof}

\ex Let $f:\R^2\to \R$ defined as $f(x,y)=x$ and $K$ be the parallelogram containing the vertices $(0,0)$, $(0,1)$, $(2,3)$, and $(2,2)$. Evaluate the integral of $f$ over $K$.

\textit{Solution:} We will make a substituion to deform the parallelogram into a square. Build a linear transformation $\varphi$ that maps $R=[0,1]\times [0,1]$ onto $K$, in particular, $\varphi$ will send $e_1\mapsto e_2$ and $e_2\mapsto 2e_1+2e_2$. Then we have that the standard matrix for $\varphi$ is
$$
A=\mat 0 & 2 \\ 1 & 2 \trix,
$$
so $\varphi^{-1}$ has standard matrix
$$
A^{-1} = \frac{1}{-2}\mat 2 & -2 \\ -1 & 0 \trix.
$$
Then under $\varphi^{-1}$,
$$
\mat x \\ y \trix \mapsto \mat -x + y \\ \frac{1}{2}x \trix
$$
and $\varphi^{-1}(K)=R,$ the unit square. Then Theorem 1 tells us
$$
\begin{aligned}
\int_R f &= \int_R (f\circ \varphi)|\det(D\varphi)|\\
&= \int_R 2y\cdot 2\\
&=\int_0^1 \int_0^1 4y ~dydx.
\end{aligned}
$$

\subsection{Polar Coordinates}
 Polar coordinates are particularly useful from $\R^2\to \R^2$ if the domain has certain radial symmetries or the integrand becomes much simpler in polar coordinates. Consider the function $\varphi$ defined as
 $$
 \mat r \\ \theta \trix \mapsto \mat r\cos\theta \\ r\sin\theta \trix
 $$
defined for $(r,\theta)$ satisfying $r>0$ and $0<\theta<2\pi$. Let $U=\{(r,\theta):r>0,0<\theta<2\pi\}$. Then $\varphi$ is a $C^1$ diffeomorphism from $U$ onto $\R^2\bslash\{(x,y):y=0, x\geq 0\}$. If we do not cut the non-negative $x$-axis, then $\varphi^{-1}$ is discontinuous along $\theta = 0$, and thus no longer a diffeomorphism. However, this is not bad nuance since the cut has measure zero.

Doing a polar coordinate substitution works by Theorem 1.


\subsection{Spherical Coordinates}

Recall the change of coordinates transformation $\psi:\R^3\to \R^3$ from spherical coordinates to Cartesian coordinates was defined as
$$
\mat \rho \\ \phi \\ \theta \trix \mapsto \mat \rho\sin\phi\cos\theta \\
\rho \sin\phi \sin\theta \\ \rho\cos\phi \trix
$$
over the domain $U=\{(\rho,\phi,\theta)\in\R^3 : \rho>0, 0<\phi<\pi , 0<\theta<2\pi\}$. Then $\psi$ is a $C^1$ diffeomorphism fro $U$ onto $\R^3\bslash \{(x,y,z):x\geq 0, y=0\}$.

For reference, $|\det(D\psi)|=\rho^2\sin\phi$.
\end{document}