\documentclass[11pt]{article}
\usepackage[utf8]{inputenc}	% Para caracteres en español
\usepackage{amsmath,amsthm,amsfonts,amssymb,amscd}
\usepackage{multirow,booktabs}
\usepackage[table]{xcolor}
\usepackage{fullpage}
\usepackage{lastpage}
\usepackage{enumitem}
\usepackage{fancyhdr}
\usepackage{mathrsfs}
\usepackage{wrapfig}
\usepackage{setspace}
\usepackage{calc}
\usepackage{stmaryrd}
\usepackage{multicol}
\usepackage{cancel}
\usepackage[retainorgcmds]{IEEEtrantools}
\usepackage[margin=3cm]{geometry}
\usepackage{amsmath}
\newlength{\tabcont}
\setlength{\parindent}{0.0in}
\setlength{\parskip}{0.05in}
\usepackage{empheq}
\usepackage{framed}
\usepackage[most]{tcolorbox}
\usepackage{xcolor}
\colorlet{shadecolor}{orange!15}
\parindent 0in
\parskip 12pt
\geometry{margin=1in, headsep=0.25in}
\theoremstyle{definition}
\newtheorem{defn}{Definition}
\newtheorem{reg}{Rule}
\newtheorem{exer}{Exercise}
\newtheorem{note}{Note}
\newtheorem{prop}{Proposition}
\newtheorem{ex}{Example}
\newcommand{\R}{\mathbb{R}}                      % Wes's shortcut command for the set of all real numbers
\newcommand{\C}{\mathbb{C}}                      % Wes's shortcut command for the set of all complex numbers
\newcommand{\Q}{\mathbb{Q}}
\newcommand{\N}{\mathbb{N}}
\usepackage{enumerate}
\renewcommand\thesection{\S\arabic{section}}
\newtheorem{theorem}{Theorem}
\newtheorem{lem}{Lemma}
\usepackage{mdframed}
\newcommand{\bslash}{\symbol{92}}
\newcommand{\upint}[1][2]{\overline{\int_{#1}^{#2}}}
\newcommand{\loint}[1][2]{\underline{\int_{#1}}^{#2}}
\usepackage{dirtytalk}
\newcommand{\bconditions}{\left\{\begin{aligned}}
\newcommand{\econditions}{\end{aligned}\right.}
\newcommand{\mat}{\begin{bmatrix}}
\newcommand{\trix}{\end{bmatrix}}

\newcommand{\dell}{\partial}
\begin{document}


\thispagestyle{empty}

\begin{center}
{\LARGE \bf Lecture 35: The Hilbert Space $\mathbf{L}^\mathbf{2}$, Orthogonal Families of Functions}\\
{\large Friday, 21 April 2023}\\
\end{center}

\section{Inner Products on Function Spaces}
Recall that if $V$ is a vector space over $\R$, an inner product $\langle \cdot,\cdot \rangle:V\times V\to \R$ obeys the following properties:
\begin{enumerate}
    \item Positive definiteness: for all $u,v\in V$, $\langle u, v\rangle \geq 0$, with equality if and only if $u=0$ or $v=0$.
    \item Symmetry: for all $u,v\in\R$, $\langle u,v\rangle = \langle v, u \rangle$.
    \item Linearity in the first slot: for any $u,v,w\in V$ and scalar $c_1\in \R$, we have that
    $$
    \langle u+v, w\rangle = \langle u,w\rangle + \langle v, w \rangle 
    $$
    and
    $$
    \langle cv , u \rangle = c\langle v,u\rangle.
    $$
\end{enumerate}
A consequence of this is that inner products are symmetric, positive definite, bilinear forms. Thus, any vector space with an inner product has a norm induced by the inner product:
$$
\|v\|=\sqrt{\langle v,v \rangle}.
$$
Norms are a generalization of distance and thus can give us a notion of convergence.
\ex Let $V$ be space of real-valued continuous functions on $[a, b]$. Then $V$ is an inner product space with the inner product
$$\langle f, g \rangle = \int_a^b f(x)g(x)~dx.$$
The inner product is well-defined as $f$ and $g$ are both continuous, so their product is continuous and $[a,b]$ is a compact set. This inner product then induces the norm
$$
\|f\| = \left(\int_a^b f(x)^2~dx\right)^\frac{1}{2}.
$$
Notice this is not the norm we have been using so far in this class over $C[a,b]$, as the norm we have been using was defined as
$$
\|f\|_\infty = \max_{x\in[a,b]}|f(x)|,
$$
which was not induced by an inner product. This new norm we have defined is called the $L^2$ norm of $f$ on $[a,b]$. 

This begs the question: is $V$ from the example above with the $L^2$ norm complete? For some functions $f,g\in V$ we have that
$$
\|f-g\| = \left( \int_a^b [f(x)-g(x)]^2 ~dx \right)^\frac{1}{2},
$$
so $V$ is not complete in the $L^2$ norm. We can see this with can example: consider the sequence of functions $\{f_n\}$ defined as
$$
f_n=\left\{ \begin{aligned}
&1, &a\leq x \leq \frac{a+b}{2}\\
&1 - \frac{x-\frac{a+b}{2}}{x_n - \frac{a+b}{2}}, &\frac{a+b}{2}< x < x_n\\
&0, & x_n \leq x\leq \frac{a+b}{2}
\end{aligned}, \qquad \forall n\in\N
\right.
$$
where 
$$x_n=\frac{a+b}{2}+\frac{b-a}{2n}.$$
Each $f_n$ is continuous and the sequence can shown to be Cauchy under the $L^2$ norm, but the sequence does not converge to a continuous function.
\section{The Hilbert Space $\mathbf{L}^\mathbf{2}$}
Now, we want to the structure of an inner product space with the property that we can represent each $f\in V$ as the linear combination
    $$
    \sum_{k=0}^\infty c_k\phi_k,
    $$
    where $\{\phi_k:k\in\N \}$ is an othonormal family of functions and can approximate $f$ as a finite linear combination.

What space of functions can we use? And what inner product accordingly? What even is meant by \say{represent} or \say{approximate}? As found earlier, we cannot use $C[a,b]$ with the $L^2$ inner product as it is not complete. Instead, we use the $L^2$ Hilbert space.

\begin{mdframed}[backgroundcolor = blue!10]
\vspace{+0.1cm}
\defn Be warned, this is not the formal definition! The space $L^2[a,b]$ is the space of \say{functions} $f:[a,b]\to \R$ such that $f^2$ is \say{integrable,} so we have that
$$
\int_a^b f(x)^2~dx<\infty.
$$
\end{mdframed}
\note In the formal definition of $L^2$ space, the elements of $L^2$ are not functions but rather \textit{equivalence classes} of functions. In this case, $f\sim g$ when $f$ and $g$ differ only on a set of measure $0$. Furthermore, integrable in this definition means Lebesgue integrable.

\ex Consider the set $\Q\cap [0,1]$. This set has measure zero. What does this mean? Let $\varepsilon>0$ be given, and let $q_1,q_2, \dots$ enumerate the rationals in $[0,1]$. Enclose $q_j$ in an interval of length $\varepsilon/2^j$ (this the the measure of the interval around $q_j$). The total length of these intervals is $\varepsilon$.

We claim that $L^2[a,b]$ is a real inner product space with the inner product
$$\langle f, g \rangle = \int_a^b f(x)g(x)~dx.$$
This would be straightforward if all members of $L^2[a,b]$ were continuous, but elements of $L^2[a,b]$ need not be bounded. 

\ex Consider the function $f:[0,1]\to \R$
$$f(x) = \begin{cases}

0 & x = 0\\

\frac{1}{x^{1/4}} & x \in (0, 1]

\end{cases}$$
We have that
$$
\int_0^1 f(x)^2 ~dx =\int_0^1 \frac{1}{\sqrt{x}} ~dx = \lim_{a\to0^+} \int_a^1 x^{-1/2} ~dx = 2,
$$
so $f\in L^2[0,1]$.
\section{Orthogonal Functions}
Recall the definition of orthogonality.
\begin{mdframed}[backgroundcolor = blue!10]
\vspace{+0.1cm}
\defn If $V$ is a real inner product space, two function $f,g\in V$ are said to be \textit{orthogonal} if $\langle f,g\rangle =0$
\end{mdframed}
\ex Consider any two functions $f,g\in L^2[-1,1]$ that are piecewise continuous. If $f$ is an odd fucntion and $g$ is an even function, then $f$ and $g$ are orthogonal. That is, $\langle f,g \rangle =0$. This is because the product of an odd function and an even function is an odd function, so when we take the inner product of $f$ and $g$, we are integrating an odd function over an interval symmetric about 0.

\begin{mdframed}[backgroundcolor = blue!10]
\vspace{+0.1cm}
\defn An \textit{orthonormal sequence} of functions $\phi_0,\phi_1,\phi_2,\dots$ in a real inner product space $V$ is a sequence satisfying
$$
    \langle \phi_i,\phi_j \rangle =\begin{cases}
        0, & i\neq j\\
        1, & i=j
    \end{cases}.
$$
\end{mdframed}

\ex In $L^2[-\pi,\pi]$, the set of functions
$$
\frac{1}{\sqrt{2\pi}},\;\frac{\sin{x}}{\sqrt{\pi}},\;\frac{\cos{x}}{\sqrt{\pi}},\;\frac{\sin{2x}}{\sqrt{\pi}},\;\frac{\cos{2x}}{\sqrt{\pi}},\;\dots
$$
\end{document}