\documentclass[11pt]{article}
\usepackage[utf8]{inputenc}	% Para caracteres en español
\usepackage{amsmath,amsthm,amsfonts,amssymb,amscd}
\usepackage{multirow,booktabs}
\usepackage[table]{xcolor}
\usepackage{fullpage}
\usepackage{lastpage}
\usepackage{enumitem}
\usepackage{fancyhdr}
\usepackage{mathrsfs}
\usepackage{wrapfig}
\usepackage{setspace}
\usepackage{calc}
\usepackage{multicol}
\usepackage{cancel}
\usepackage[retainorgcmds]{IEEEtrantools}
\usepackage[margin=3cm]{geometry}
\usepackage{amsmath}
\newlength{\tabcont}
\setlength{\parindent}{0.0in}
\setlength{\parskip}{0.05in}
\usepackage{empheq}
\usepackage{framed}
\usepackage[most]{tcolorbox}
\usepackage{xcolor}
\colorlet{shadecolor}{orange!15}
\parindent 0in
\parskip 12pt
\geometry{margin=1in, headsep=0.25in}
\theoremstyle{definition}
\newtheorem{defn}{Definition}
\newtheorem{reg}{Rule}
\newtheorem{exer}{Exercise}
\newtheorem{note}{Note}
\newtheorem{prop}{Proposition}
\newtheorem{ex}{Example}
\newcommand{\R}{\mathbb{R}}                      % Wes's shortcut command for the set of all real numbers
\newcommand{\C}{\mathbb{C}}                      % Wes's shortcut command for the set of all complex numbers
\newcommand{\Q}{\mathbb{Q}}
\newcommand{\N}{\mathbb{N}}
\usepackage{enumerate}
\renewcommand\thesection{\S\arabic{section}}
\newtheorem{theorem}{Theorem}
\newtheorem{lem}{Lemma}
\begin{document}


\thispagestyle{empty}

\begin{center}
{\LARGE \bf Lecture 1: The Real Numbers, Sequences, Convergence, and Completeness}\\
{\large Monday, 23 January 2023}\\

\end{center}
\section{Introduction to the Class}
\subsection{Course Goals}
\begin{enumerate}
    \item Rigorous intro to the underpinning of calculus, including multivariate.
    \item Build stamina for longer proofs.
    \item Become independent in reading technical mathematics.
\end{enumerate}
\textit{Question:} How do we begin?

\section{Entry Points: Why $\R$?}
\subsection{Approaches to $\R$}
We begin with another question: what should be axiomatic?  Why develop a calculus for functions that are real-valued ($\R\to\R$) as opposed to functions over the rational numbers ($\mathbb{Q}\to\mathbb{Q}$)? One approach is to build the number system from the ground up.

\ex Let us build the natural numbers. We can take as axiomatic the existence of the empty set $\emptyset$. We identify this with 0. Then we define a successor function $S(n)=n\cup \{n\}$. Then we have
$$
S(0)=\emptyset \cup \{\emptyset\} = \{\emptyset\}.
$$
We will idenitify this set with 1 and define 1 then as the cardinality of this set. Then we have
$$
S(1)=\{\emptyset\}\cup \{\{\emptyset\}\} =\{\emptyset,\{\emptyset\}\}.
$$
We will define 2 as the cardinality of this set. So on and so forth. Now to define a function $f:\mathbb{N}\times\mathbb{N}\to \mathbb{N}$ that constitutes addition...

Notice that this takes quite a bit of time and seems more appropriate for a different class, algebra or set theory. If you want to see this approach, the construction of $\R$ from $\mathbb{Q}$, refer to section 8.6 of Abbott.

Let us approach from a different angle. Instead, we will begin with $\R$ already begin constructed and adopt the following theorem:
\theorem (\textit{Axiom of completeness}) Every non-empty subset of $\R$ that is bounded above has a least upper bound (a \textit{supremum}). While this is a theorem we can prove from more fundamental pieces, we will accept this as an axiom.

This is motivated by the fact that a big part of calculus deals with the notion of a limit. We want to know that a sequence that is bounded above and is monotonically increasing will converge to something.

\ex Consider the sequence
$$
1, 1.4, 1.41, 1.414,\dots
$$
in $\mathbb{Q}$. It is certainly bounded above by 2 and monotone increasing. Could we say that this sequence converges in $\Q$? No! The value it "approaches," $\sqrt{2}$, does not exist in $\Q$ so it does not converge to anything in $\Q$. In other words,
$$
S=\{x\in\Q\;|\; x^2<2\}
$$
has no supremum in $\Q$. However, there is no issue in $\R$ of having these "gaps" which is helpful in defining things like compactness and continuity.

\subsection{The Archimedean Property}
\prop (\textit{The Archimedean Property}) Given any $x\in\R$, there exists an $n\in \mathbb{N}$ such that $n>x$.
\begin{proof}
    Suppose that there exists some $x\in\R$ such that for all $n\in \N$, $n\leq x$. Then $\N$ is non-empty and bounded above by $x$, so $\N$ has a supremum $s\in\R$ by the axiom of completeness. Then $s-1$ is \textit{not} an upper bound for $\N$ and so there exists an $n\in N$ with $n>s-1$. Then $s<n+1\in\N$, which contradicts that $s$ was a supremum.
\end{proof}
"This innocent result has a beautiful corollary" -Wes

\textit{Corollary:} Given any $\varepsilon>0$ in $\R$, there exists $n\in\N$ such that $1/n<\varepsilon$.
\begin{proof}
    Use $x-1/\varepsilon$ in Archemedean property.
\end{proof}
We get a lot of mileage out of this property!
\section{Sequences and Convergence}
\defn A \textit{sequence} on a set $S$ is a function $f:\N\to S$. By convention, we use subscripts instead of the function notation to denote this (that is, we use $x_n$ instead of $f(n)$). The whole sequence can be denoted $\{x_n\}_{n=1}^\infty = \{x_n\}$.
\ex Consider the sequence
$$
x_n=\frac{n}{n^2+1}.
$$
We will notate this as 
$$
\left\{\frac{n}{n^2+1}\right\}.
$$
\defn Let $\{x_n\}$ be a sequence in $\R$. We say that $\{x_n\}$ \textit{converges} to a \textit{limit} $L$ if for all $\varepsilon>0$, there exists an $N\in\N$ such that $|x_n-L|<\varepsilon$ for all $n\geq N$.
\ex Let $x_n=1/n$. Prove that  $\{x_n\}\to 0$.

\textit{Solution:} Given any $\varepsilon>0$, choose an $N\in \N$ such that $1/N<\varepsilon$, which we know exists by the Archimedean Property. Then for all $n\in \N$ with $n\geq N$,
$$
|x_n-0|=\left|\frac{1}{n}\right|=\frac{1}{n}\leq \frac{1}{N}<\varepsilon.
$$

Can a sequence have multiple convergence points?
\prop If a sequence in $\R$ converges, the limit is unique.

\begin{proof}
    Suppose $\{x_n\}\to L_1$ and $\{x_n\}\to L_2$. Since $\{x_n\}\to L_1$, we know that for all $\varepsilon>0$, there exists an $N_1\in\N$ such that $|x_n-L_1|<\varepsilon/2$ for all $n>N_1$. Furthermore, since $\{x_n\}\to L_2$, we know that for all $\varepsilon>0$, there exists an $N_2\in\N$ such that $|x_n-L_2|<\varepsilon/2$ for all $n>N_2$. Now, let $N=\max \{N_1,N_2\}$. Then for all $n\geq N$, we have
    $$
    \begin{aligned}
    |L_1-L_2|&=|L_1-x_n+x_n-L_2|\\
    &\leq |L_1-x_n|+|x_n-L_2|\\
    &<\frac{\varepsilon}{2} +\frac{\varepsilon}{2}=\varepsilon.
    \end{aligned}
    $$
    Then $L_1-L_2=0$ so $L_1=L_2$.
\end{proof}

\end{document}