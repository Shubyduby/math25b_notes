\documentclass[11pt]{article}
\usepackage[utf8]{inputenc}	% Para caracteres en español
\usepackage{amsmath,amsthm,amsfonts,amssymb,amscd}
\usepackage{multirow,booktabs}
\usepackage[table]{xcolor}
\usepackage{fullpage}
\usepackage{lastpage}
\usepackage{enumitem}
\usepackage{fancyhdr}
\usepackage{mathrsfs}
\usepackage{wrapfig}
\usepackage{setspace}
\usepackage{calc}
\usepackage{multicol}
\usepackage{cancel}
\usepackage[retainorgcmds]{IEEEtrantools}
\usepackage[margin=3cm]{geometry}
\usepackage{amsmath}
\newlength{\tabcont}
\setlength{\parindent}{0.0in}
\setlength{\parskip}{0.05in}
\usepackage{empheq}
\usepackage{framed}
\usepackage[most]{tcolorbox}
\usepackage{xcolor}
\colorlet{shadecolor}{orange!15}
\parindent 0in
\parskip 12pt
\geometry{margin=1in, headsep=0.25in}
\theoremstyle{definition}
\newtheorem{defn}{Definition}
\newtheorem{reg}{Rule}
\newtheorem{exer}{Exercise}
\newtheorem{note}{Note}
\newtheorem{prop}{Proposition}
\newtheorem{ex}{Example}
\newcommand{\R}{\mathbb{R}}                      % Wes's shortcut command for the set of all real numbers
\newcommand{\C}{\mathbb{C}}                      % Wes's shortcut command for the set of all complex numbers
\newcommand{\Q}{\mathbb{Q}}
\newcommand{\N}{\mathbb{N}}
\usepackage{enumerate}
\renewcommand\thesection{\S\arabic{section}}
\newtheorem{theorem}{Theorem}
\newtheorem{lem}{Lemma}
\usepackage{mdframed}
\newcommand{\bslash}{\symbol{92}}

\begin{document}


\thispagestyle{empty}

\begin{center}
{\LARGE \bf Lecture 12: Key Results Related to Differentiation and Optimization}\\
{\large Friday, 17 February 2023}\\

\end{center}


\section{Differentiation and Extrema}

\prop If $f:\Omega\to \R$ where $\Omega\subseteq \R$ and $f$ has a local extremum at $x_0\in\Omega$. Then $f'(x_0)=0$.

\begin{proof}
    Suppose $f$ has a local maximum at $x_0$. Assume $f'(x_0)$ exists and is nonzero. Assme that $f(x_0)>0$. For all $\varepsilon>0$, there exists an $\delta>0$ such that if $0<|x-x_0|<\delta$, then
    $$
    \left|\frac{f(x)-f(x_0)}{x-x_0}-f'(x_0)\right|<\varepsilon.
    $$
    Set $\varepsilon=f'(x_0)/2>0$. Then we have
    \begin{align*}
        -\varepsilon&<\frac{f(x)-f(x_0)}{x-x_0}-f'(x_0)\\
        \frac{1}{2}f'(x_0)&<\frac{f(x)-f(x_0)}{x-x_0}\\
        f(x)&>f(x_0)+\frac{1}{2}f'(x_)(x-x_0).
    \end{align*}
    If $x>x_0$, we get $f(x)>f(x_0)$, a contradiction.
\end{proof}

\textit{Corollary 1.} If $I$ is an interval and $f:I\to\R$ is continuous on $I$ and $f$ has a local extremum at some interior point $c\in I$, then either $f'(c)=0$ or $f'(c)$ does not exist.

\ex The function $f(x)=|x|$ on $[-1,1]$ has an undefined derivative at the minimum.

\section{The Mean Value Theorem}

\begin{shaded}
\theorem (\textit{Rolle}) Suppose $f:[a,b]\to \R$ is continuous and is differentiable on $(a,b)$, and that $f(a)=f(b)=0$. Then there exists a $c\in(a,b)$ such that $f'(c)=0$.
\end{shaded}

\begin{proof}
    If $f=0$ on $[a,b]$, then this is straightforward as $f'(c)=0$ at every interior point $c$.

    Otherwise, without loss of generality, assume that $f$ takes on a positive value somewhere in its domain. By the extreme value theorem, there exists some $c\in[a,b]$ such that $f(c)=\sup\{f(x):x\in[a,b]\}$. So $f$ achieves an absolute and therefore local extremum at $c$. Then, by proposition 1, $f'(c)=0$. 
\end{proof}

\textit{Corollary 2.} (\textit{Mean Value Theorem}) If $f$ is continuous on $[a,b]$ and differentiable on the interior $(a,b)$, then there exists some $c\in(a,b)$ such that 
$$ 
f'(c)=\frac{f(b)-f(a)}{b-a}.
$$
\begin{proof}
    Let 
    $$
    L(x)=f(a)+\frac{f(b)-f(a)}{b-a}(x-a).
    $$
    Consider $g(x)=f(x)-L(x)$. Then $g(a)=g(b)=0$ and $g$ is differentiable on $(a,b)$ since $f$ and $L$ are. Then Rolle's theorem.
\end{proof}

\textit{Corollary 3.} If $f$ continuous on $[a,b]$ and differentiable on $(a,b)$ and $f'(x)=0$ on $[a,b]$, then $f$ is constant on $[a,b]$.

\begin{proof}
    Let $a<x\leq b$. Then $f$ is continuous $[a,x]$ and differentiable $(a,x)$. By the mean value theorem, we can choose $c\in(a,x)$ with 
    $$
    f'(c)=\frac{f(x)-f(a)}{x-a}.
    $$
    But $f'(c)=0$, so $f(x)=f(a)$.
\end{proof}

\textit{Corollary 4.} If $f$ and $g$ are continuous on $[a,b]$ and differentiable on $(a,b)$ and $f'(x)=g'(x)$ on $(a,b)$, then there exists a constant $C$ such that $f(x)=g(x)+C$ for all $x\in[a,b]$.
\begin{proof}
    Apply Corollary 3 to the function $f-g$. 
\end{proof}

\section{Optimization}
We are going to set up the 1st derivative test.
\prop Let $f$ be continuous on the closed interval $[a,b]$ and differentiable on the interval $(a,b)$. Then $f$ is increasing on $[a,b]$ if and only if $f'(x)\geq 0$ for all $x\in(a,b)$, and decreasing on $[a,b]$ if and only if $f'(x)\leq 0$ for all $x\in (a,b)$. 
\begin{proof}
    Let us prove the case of increasing $f$. Assume that $f'(x)\geq0$ on $(a,b)$. Pick any two points $x_1,x_2\in[a,b]$ and $x_1<x_2$. By the mean value theorem, there exists some $c\in(x_1,x_2)$ such that 
    $$
    f'(c)=\frac{f(x_2)-f(x_1)}{x_2-x_1}.
    $$
    Then we have $f(x_2)=f(x_1)+\underbrace{(x-2)(x_1)f'(c)}_{\geq 0}$.

    Now suppose that $f$ is differentiable on $(a,b)$ and increasing on $[a,b]$. Pick any $x_0\in (a,b)$. Then given any $x\neq x_0$ in $(a,b)$, 
    $$
    \frac{f(x)-f(x_0)}{x-x_0}\geq 0.
    $$
    So we have
    $$
    \lim_{x\to x_0}\frac{f(x)-f(x_0)}{x-x_0}=f'(x_0)\geq 0.
    $$
\end{proof}

\begin{shaded}
    \theorem (\textit{1st Derivative Test}) Let $f$ be continuous on $[a,b]$ and for some $c\in(a,b)$, $f$ is differentiable from $(a,c)$ and $(c,b)$. Furthermore, suppose there exists some $\delta>0$ such that 
    \begin{enumerate}
        \item $(c-\delta,c+\delta)\subseteq (a,b)$
        \item $f'(x)\geq 0$ on $(c-\delta, c)$
        \item $f'(x)\leq 0$ on $(c,c+\delta)$.
    \end{enumerate}
    Then $c$ is a local maximum on the interval $[a,b]$. There is an analgous result of local minima.
\end{shaded}
\begin{proof}
    If $x\in(c-\delta,c)$, then by the mean value theorem, we can choose some $\tilde{x}\in(x,c)$ such that $f(c)-f(x)=\underbrace{f'(\Tilde{x})}_{\geq 0}\cdot \underbrace{(c-x)}_{>0}$. So $f(c)\geq f(x)$ for all $x$. Same can we done with other half. 
\end{proof}
\note The converse is not true! Consider the function $f:\R\to\R$ defined as
$$
f(x)\left\{\begin{aligned}
    &2x^4+x^4\sin(1/x),& x\neq 0\\
    &0,&x=0
\end{aligned}\right..
$$
Check that $f$ has an absolution minimum at $x=0$ but $f'(x)$ takes on both positive and negative values in all neighborhoods of $x$.

\begin{shaded}
    \theorem (\textit{Inverse Function Theorem}) Suppose $f:(a,b)\to \R$ is differentiable and either $f'(x)>0$ for all $x\in(a,b)$ or $f'(x)<0$ for all $x\in (a,b)$. Then $f$ is a bijection from $(a,b)$ onto its range, and $f^{-1}$ is differentiable on its domain, and if $y=f(x)$, then $(f^{-1})'(y)=1/f'(x)$.
\end{shaded}
Another theorem!! Three theorems in one class?!? This is unheard of! Find the proof in the textbook.
\end{document}