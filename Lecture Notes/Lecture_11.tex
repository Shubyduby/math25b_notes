\documentclass[11pt]{article}
\usepackage[utf8]{inputenc}	% Para caracteres en español
\usepackage{amsmath,amsthm,amsfonts,amssymb,amscd}
\usepackage{multirow,booktabs}
\usepackage[table]{xcolor}
\usepackage{fullpage}
\usepackage{lastpage}
\usepackage{enumitem}
\usepackage{fancyhdr}
\usepackage{mathrsfs}
\usepackage{wrapfig}
\usepackage{setspace}
\usepackage{calc}
\usepackage{multicol}
\usepackage{cancel}
\usepackage[retainorgcmds]{IEEEtrantools}
\usepackage[margin=3cm]{geometry}
\usepackage{amsmath}
\newlength{\tabcont}
\setlength{\parindent}{0.0in}
\setlength{\parskip}{0.05in}
\usepackage{empheq}
\usepackage{framed}
\usepackage[most]{tcolorbox}
\usepackage{xcolor}
\colorlet{shadecolor}{orange!15}
\parindent 0in
\parskip 12pt
\geometry{margin=1in, headsep=0.25in}
\theoremstyle{definition}
\newtheorem{defn}{Definition}
\newtheorem{reg}{Rule}
\newtheorem{exer}{Exercise}
\newtheorem{note}{Note}
\newtheorem{prop}{Proposition}
\newtheorem{ex}{Example}
\newcommand{\R}{\mathbb{R}}                      % Wes's shortcut command for the set of all real numbers
\newcommand{\C}{\mathbb{C}}                      % Wes's shortcut command for the set of all complex numbers
\newcommand{\Q}{\mathbb{Q}}
\newcommand{\N}{\mathbb{N}}
\usepackage{enumerate}
\renewcommand\thesection{\S\arabic{section}}
\newtheorem{theorem}{Theorem}
\newtheorem{lem}{Lemma}
\usepackage{mdframed}
\newcommand{\bslash}{\symbol{92}}

\begin{document}


\thispagestyle{empty}

\begin{center}
{\LARGE \bf Lecture 11: Differentiation}\\
{\large Wednesday, 15 February 2023}\\

\end{center}
\section{Differentiability}
We can now describe what it means for a function to be differentiable. Here is the general notion taught by most first calculus classes.\:
\begin{mdframed}[backgroundcolor = blue!10]

\vspace{+0.2cm}

\defn Let $f:\Omega\subseteq \R\to \R$ where $\Omega$ is open. We say that $f$ is \textit{differentiable} at $x_0\in\Omega$ if 
$$
\lim_{x\to x_0}\frac{f(x)-f(x_0)}{x-x_0}
$$
 exists. The limit will be denoted as $f'(x_0)$ and called the \textit{derivative} of $f$ at $x_0$.
\end{mdframed}
\note Since $\Omega$ is open and $x_0\in\Omega$, $x_0$ is a limit point.

\ex Let $\Omega=(0,\infty)$ and $f:\Omega\to \R$ be $f(x)=\sqrt{x}$. Find the derivative of $f$ at $x_0$.

\textit{Solution:} Pick any $x_0\in(0,\infty)$. We claim that $f'(x_0)$ exists. Observe,
$$
\lim_{x\to x_0}\frac{\sqrt{x}-\sqrt{x_0}}{x-x_0}\cdot\frac{\sqrt{x}+\sqrt{x_0}}{\sqrt{x}+\sqrt{x_0}}=\lim_{x\to x_0}\frac{1}{\sqrt{x}+\sqrt{x_0}}=\frac{1}{2\sqrt{x_0}}.
$$
The last equality holds because of the continuity of $f$ at $x_0$. Indeed, we know the square-root function is continuous and thus $\lim_{x\to x_0}\sqrt{x}=\sqrt{x_0}$.

We have a proposition that stems from this idea.

\prop If $f:\Omega\subseteq \R\to\R$ is differentiable at $x_0$, then $f$ is continuous at $x_0$.

\begin{proof}
    Assume that $f'(x_0)$ exists. We want to show that $\lim_{x\to x_0} f(x)=f(x_0)$. Notice that
    $$
    \lim_{x\to x_0} f(x)=\lim_{x\to x_0} \left[\left(\frac{f(x)-f(x_0)}{x-x_0} \right)(x-x_0)+f(x_0)\right]=f'(x_0)\cdot 0 +f(x_0).
    $$
    The last equality holds since the product of limits is equal to the limits of products.
\end{proof}
\note Continuity at $x_0$ does not imply differentiability.
\ex $f:\R\to\R$ defined as $f(x)=|x|$ is not differentiable at $x=0$.

\ex $g:\R\to\R$ defined as $g(x)=\sqrt[3]{x}$ is not differentiable at $x=0$.

\ex The piecewise function $h$ defined as 
$$
h(x)\left\{\begin{aligned}
&x\sin{\left(\frac{1}{x}\right)},&x\neq0\\
&0,&x=0.
\end{aligned}\right.
$$
is continuous but not differentiable at 0. We can show that it is continuous. Let $\varepsilon>0$ and choose $\delta=\varepsilon$. Then when $|x-0|<\delta$, we have 
$$
|f(x)-h(0)|=|h(x)-0|=\left\{\begin{aligned}
    &\left|x\sin{\left(\frac{1}{x}\right)}\right|\leq x,&x\neq 0\\
    &0,&x=0
\end{aligned}\right.<\varepsilon.
$$
Using normal calculus you can find that it is indeed not differentiable at 0.

\section{Differentiation Rules}
Suppose $f:\Omega\subseteq \R\to\R$ and $g:\Omega\subseteq \R\to \R$, where $\Omega$ is open. We have $x_0\in\Omega$ and $c\in \R$. If $f,g$ are differentiable at $x_0$, then so are $f+g$, $f-g$, $cf$, and $fg$. So is $f/g$, provided that $g(x_0)\neq 0$.

\subsection{Sum Rule}
We prove that the sum rule.
\begin{proof}
    Let $\varepsilon>0$ be given. Since $f$ is differentiable at $x_0$, there exists a $\delta>0$ such that if $0<|x-x_0|<\delta_1$, we have that 
    $$
    \left|\frac{f(x)-f(x_0)}{x-x_0}-f'(x_0)\right|<\varepsilon.
    $$
    Then if we multiply both sides by $|x-x_0|$,
    $$
    |f(x)-f(x_0)-f'(x_0)(x-x_0)|\leq\varepsilon |x-x_0|.
    $$
    The equation above is \textbf{the equivalent} to being differentiable at $x_0$. Ponder what it is saying! We can do the same with $g$. Find a $\delta_2$ such that $|x-x_0|<\delta_2$ implies
    $$
    |g(x)-g(x_0)-g'(x_0)(x-x_0)|\leq\varepsilon |x-x_0|.
    $$
    Let $\delta=\min\{\delta_1,\delta_2\}>0$. Then whenever $|x-x_0|<\delta$, we have
    $$
    |(f+g)(x)-(f+g)(x_0)-[f'(x_0)+g'(x_0)](x-x_0)|\leq 2\varepsilon|x-x_0|
    $$
    by the triangle inequality.
\end{proof}
\note A consequence of this is that $(f+g)'(x_0)=f'(x_0)+g'(x_0)$.

\subsection{Product Rule}
Recall the product rule:
$$
(fg)'(x_0)=f'(x_0)g(x_0)+f(x_0)g'(x_0).
$$
You will prove this on this week's problem set. Here is the heuristic proof:

We have
$$
\begin{aligned}
\lim_{h\to 0}\frac{(fg)(x_0+h)-(fg)(x_0)}{h}&=\lim_{h\to 0}\frac{f(x_0+h)g(x_0+h)-f(x_0)g(x_0)}{h}\\
&= \lim_{h\to 0}\frac{f(x_0+h)g(x_0+h)-f(x_0+h)g(x_0)+f(x_0+h)g(x_0)-f(x_0)g(x_0)}{h}\\
&=\lim_{h\to 0}\left\{f(x_0+h)\left(\frac{g(x_0+h)-g(x_0)}{h}\right)+g(x_0)\left(\frac{f(x_0+h)-f(x_0)}{h}\right)\right\}\\
&=f(x_0)g'(x_0)+g(x_0)f'(x_0).
\end{aligned}
$$

\subsection{Quotient Rule}
Let $r(x)=p(x)/q(x)$, where $p(x)$ and $q(x)$ are differentiable. Rearrange into $p(x)=r(x)q(x)$ and use the product rule. Then solve for $r'(x)$.

\subsection{Chain Rule}
Let $f:\Omega_1\subseteq \R\to\R$ and $g:\Omega_2\subseteq \R\to\R$, where $\Omega_1$ and $\Omega_2$ are open sets and $f(\Omega_1)\subseteq\Omega_2$. If $f$ is differentiable at $x_0\in\Omega_1$ and $g$ differentiable at $f(x_0)\in\Omega_2$, then $g\circ f$ is differentiable at $x_0$ and $(g\circ f)'(x_0)=g'(f(x_0))\cdot f'(x_0)$. Wes will post the proof for the chain rule on Canvas.

\section{Continuity of the Derivative}
Recall that $f(x)=\sqrt{x}$ and $x_0>0$, we have $f'(x_0)=1/(2\sqrt{x_0})$. It is a function in its own right. Will it always be continuous though?

\ex If $f$ is differentiable, $f'$ need not be continuous! Consider
$$
f(x)\left\{\begin{aligned}
&x^2\sin{\left(\frac{1}{x}\right)},&x\neq0\\
&0,&x=0.
\end{aligned}\right.
$$
We have that 
$$
\lim_{h\to 0}\frac{f(h)-f(0)}{h}=\lim_{h\to 0}\frac{h^2\sin(1/h)}{h}=\lim_{h\to 0}h\sin(1/h)=0=f'(0).
$$

If $x\neq 0$, we have $f'(x)=2x\sin(1/x)-\cos(1/x)$. This function has pretty wacky behavior as it approaches the origin. Confirm to yourself the function is not continuous at $x=0$.

\section{Terminology}
Let $f:\Omega\subseteq \R\to \R$, where $\Omega$ is open. 

\defn Let $I$ be a nontrivial interval $I\subseteq \Omega$. Say $f$ is \textit{increasing on} $I$ if given $x,y\in I$ with $x<y$ we have $f(x)\leq f(y)$. We say $f$ has a \textit{local maximum} at $x_0\in\Omega$ if there exists $\varepsilon>0$ with $(x_0-\varepsilon,x_0+\varepsilon)\subseteq \Omega$ and $f(x_0)\geq f(x)$ for all $x$ in the $\varepsilon$ interval.

\note Just because the derivative is positive at a single point, it does not mean that it is increasing over an interval. Marsden and Hoffman use the term "increasing at a point" to refer to a positive having a positive derivative but it is \textit{not} equivalent to increasing over an interval.
\end{document}
