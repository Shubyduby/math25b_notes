\documentclass[11pt]{article}
\usepackage[utf8]{inputenc}	% Para caracteres en español
\usepackage{amsmath,amsthm,amsfonts,amssymb,amscd}
\usepackage{multirow,booktabs}
\usepackage[table]{xcolor}
\usepackage{fullpage}
\usepackage{lastpage}
\usepackage{enumitem}
\usepackage{fancyhdr}
\usepackage{mathrsfs}
\usepackage{wrapfig}
\usepackage{setspace}
\usepackage{calc}
\usepackage{multicol}
\usepackage{cancel}
\usepackage[retainorgcmds]{IEEEtrantools}
\usepackage[margin=3cm]{geometry}
\usepackage{amsmath}
\newlength{\tabcont}
\setlength{\parindent}{0.0in}
\setlength{\parskip}{0.05in}
\usepackage{empheq}
\usepackage{framed}
\usepackage[most]{tcolorbox}
\usepackage{xcolor}
\colorlet{shadecolor}{orange!15}
\parindent 0in
\parskip 12pt
\geometry{margin=1in, headsep=0.25in}
\theoremstyle{definition}
\newtheorem{defn}{Definition}
\newtheorem{reg}{Rule}
\newtheorem{exer}{Exercise}
\newtheorem{note}{Note}
\newtheorem{prop}{Proposition}
\newtheorem{ex}{Example}
\newcommand{\R}{\mathbb{R}}                      % Wes's shortcut command for the set of all real numbers
\newcommand{\C}{\mathbb{C}}                      % Wes's shortcut command for the set of all complex numbers
\newcommand{\Q}{\mathbb{Q}}
\newcommand{\N}{\mathbb{N}}
\usepackage{enumerate}
\renewcommand\thesection{\S\arabic{section}}
\newtheorem{theorem}{Theorem}
\newtheorem{lem}{Lemma}
\usepackage{mdframed}
\newcommand{\bslash}{\symbol{92}}
\newcommand{\upint}[1][2]{\overline{\int_{#1}^{#2}}}
\newcommand{\loint}[1][2]{\underline{\int_{#1}}^{#2}}
\usepackage{dirtytalk}
\newcommand{\bconditions}{\left\{\begin{aligned}}
\newcommand{\econditions}{\end{aligned}\right.}
\newcommand{\mat}{\begin{bmatrix}}
\newcommand{\trix}{\end{bmatrix}}
\newcommand{\dell}{\partial}
\begin{document}


\thispagestyle{empty}

\begin{center}
{\LARGE \bf Lecture 24: Higher Order Derivatives, Taylor-Lagrange}\\
{\large Monday, 27 March 2023}\\
\end{center}

Much of today's lecture is without proof as they are extremely technical but not unlike previous derivations.

\section{Higher Derivatives, Continued.}

Recall that if $B:\R^n\times\R^n\to \R$ is a bilinear form and $A_{ij}=B(e_i,e_j)$, then $B(x,y)=x^TAy$. 

Let $f:\Omega\subseteq \R^n\to \R$, with $\Omega$ open, and assume that $\dell f/\dell x_i$ exists throughout $\Omega$. We have that $\dell f/\dell x_i:\Omega\subseteq \R^n\to \R$. Then partial derivatives of $\dell f/\dell x_i$ may also exist. We use the notation
$$
\frac{\dell}{\dell x_j}\left(\frac{\dell f}{\dell x_i}\right)=\frac{\dell^2 f}{\dell x_j\dell x_i}
$$
or $(f_{x_i})_{x_j}=f_{x_ix_j}$. If $i=j$, we can write $\dell^2 f/\dell x_i^2$.

For functions $f:\Omega\subseteq \R^n\to \R$ which have a second derivative on $\Omega$, we can compute with second derivatives much the same way we computed first derivatives and Jacobians. We have that $D^2f(x)\in \mathcal{L}(\R^n,\mathcal{L}(\R^n,\R))$. Recall that our notation is $D^2f(x)(y,z)$ while Marsden and Hoffman use $B_x(y,z)$.

\prop Suppose that $f:\Omega\subseteq \R^n\to\R$ is twice differentiable and let $B_x(y,z)=D^2 f(x)(y,z)$. Then with respect to the standard basis, the matrix associated with this bilinear form is the following:
$$
\mat 
\frac{\dell^2 f}{\dell x_1^2} & \frac{\dell^2 f}{\dell x_1\dell x_2}&\cdots& \frac{\dell^2 f}{\dell x_1\dell x_n}\\
\frac{\dell^2 f}{\dell x_2 \dell x_1}&\ddots & &\vdots\\
\vdots &&\ddots &\vdots \\
\frac{\dell ^2 f}{\dell x_n\dell x_z}&\cdots &\cdots &\frac{\dell^2 f}{\dell x_n^2}

\trix.
$$
If $B_x:\R^n\times\R^n\to \R$ and $H$ denotes this matrix, then $B_x(y,z)=y^THz$ for all $y,z\in\R^n$. This matrix is called the \textit{Hessian}.


\prop (\textit{Clairaut-Schwartz}) If $f:\Omega\subseteq \R^n\to \R$ is twice differentiable on the open set $\Omega$ and $D^2f$ is continuous on $\Omega$, then the bilinear form $B_x(y,z)=D^2f(x)(y,z)$ is symmetric at each $x\in\Omega$, that is, $D^2(x)(y,z)=D^2f(x)(z,y)$.

\textit{Corollary.} Under the above assumptions, mixed partials are equal.

\begin{proof}
    If the matrix is symmetric, then
    $$
    \frac{\dell^2 f}{\dell x_i\dell x_j}(x)=B_x(e_i,e_j)=B_x(e_j,e_i)=\frac{\dell^2 f}{\dell x_j\dell x_i}(x).
    $$
\end{proof}

\note In calculus books, one will often see a variation of this result: if $\dell^2 f/(\dell x_i\dell x_j)$ and $\dell^2 f/(\dell x_j \dell x_i)$ exists and is continuous throughout $\Omega$, then these mixed partials are equal throughout $\Omega$.

\note For homework, we have an example of non-equality of mixed partials.

\section{Taylor-Lagrange Theorem}

\begin{mdframed}[backgroundcolor = blue!10]
\vspace{+0.1cm}

\defn A function is of class $C^r$ on its domain if its first $r$ derivatives exist and are continuous. We say $f$ is $C^\infty$ if $f\in C^r$ for all $r\in\N$. Marsden and Hoffman say \textit{smooth} to mean $C^\infty$. Some others author just require $C^1$.

\end{mdframed}
We now generalize Taylor's theorem to higher dimensions.

\begin{shaded}

\theorem(\textit{Taylor-Lagrange}) Let $f:\Omega\subseteq \R^n \to \R$ be of class $C^{k+1}$ on the open domain $\Omega$. Suppose $x_0\in\Omega$ is fixed and pick $x\in\Omega$ and assume $L(t)=x_0+t(x-x_0)\in \Omega$ for all $t\in[0,1]$. Then there exists a $t\in[0,1]$ such that
$$
\begin{aligned}
    f(x)=&\\
f(x_0)&+Df(x_0)(x-x_0)+\frac{1}{2!}D^2f(x_0)(x-x_0,x-x_0)+\cdots \\
&+ \frac{1}{k!}D^{(k)}f(x_0)\underbrace{(x-x_0,\dots,x-x_0)}_{k\mbox{ times}}+\frac{1}{(k+1)!}D^{(k+1)}f(c)\underbrace{(x-x_0,\dots,x-x_0)}_{k+1\mbox{ times}}
\end{aligned}
$$
where $c=L(t_0)$.
\end{shaded}
\note For careful book keeping, remember that $D^{(m)}f(x_0)$ is an $m$-linear map applied to $\underbrace{(x-x_0,\dots,x-x_0)}_{m\mbox{ times}}$.
\note If we write $x=x_0+h$ where $h\in\R^n$, then 
$$
f(x_0+h)=f(x_0)+Df(x_0)h+\cdots+\frac{1}{k!}D^{(k)}f(x_0)(h,\dots,h)+E(x_0,h),
$$
where $E(x_0,h)$ is the error term. The error term actually satisfies
$$
\lim_{h\to 0}\frac{\|E(x_0,h)\|}{\|h\|^k}=0.
$$
To see an example of this, let us observe the special case of $f:\R\to \R$. We can express $f(x)$ as
$$
f(x)=f(x_0)+f'(x_0)(x-x_0)+\cdots+\frac{1}{k!}f^{(k)}(x_0)(x-x_0)^k+\frac{f^{(k+1)}(c)}{(k+1)!}(x-x_0)^{k+1}.
$$
Consider the interval $I$ with endpoints at $x$ and $x_0$. Since $f^{(k+1)}$ is continuous on $I$, we have that by the extreme value theorem,
$$
M=\max_{y\in I}|f^{(k+1)}(y)|,
$$
so the error term shrinks.

\ex Consider the function $f:\R^2\to \R$ refined as 
$$
f(x,y)=\sin x+\cos y +xy.
$$
The Taylor expansion about the function at $(0,0)$ is
$$
f(x,y)=f(0,0)+\nabla f(0,0)\mat x \\ y\trix +\frac{1}{2!}D^2 f(0,0)\left(\mat x \\ y \trix, \mat x \\ y \trix\right)+\mbox{ higher order error term}.
$$
Looking at the partial derivatives, we have
$$
\begin{aligned}
f_x(x,y)&=\cos x +y\\
f_y(x,y)&=-\sin y +x
\end{aligned}
$$
So the tangent plane at $(0,0,f(0,0))$ is
$$
L(x,y)=1+\mat 1 & 0 \trix \mat x \\ y \trix = 1+x.
$$
Now, the second derivatives are
$$
\begin{aligned}
    f_{xx}(x,y)&=-\sin x\\
    f_{xy}(x,y)&=1=f_{yx}(x,y)\\
    f_{yy}(x,y)&=-\cos y.
\end{aligned}
$$
So the Hessian at $(0,0)$ is
$$
\mat f_{xx}(0,0) & f_{xy} (0,0) \\ f_{yx}(0,0) & f_{yy}(0,0)\trix=\mat 0 & 1 \\ 1 & -1 \trix.
$$
Then 
$$
\begin{aligned}
    f(x,y)&=f(0,0)+\nabla f(0,0)\mat x \\ y\trix +\frac{1}{2!}D^2 f(0,0)\left(\mat x \\ y \trix, \mat x \\ y \trix\right)+\mbox{ stuff}\\
    &=(1+x)+\frac{1}{2!}\mat x & y \trix \mat 0 & 1 \\ 1 & -1 \trix\mat x\\ y\trix+\mbox{ stuff}\\
    &=1+x+\frac{1}{2!}(xy+xy-y^2)+\mbox{ stuff}\\
    &=1+x+xy-\frac{y^2}{2!}+\mbox{ stuff},
\end{aligned}
$$
so $1+x+xy-\frac{y^2}{2!}$ is our best quadratic estimation of $f(x,y)$. Notice connected this is to the Taylor series of $\sin y$ and $\cos x$.

\begin{mdframed}[backgroundcolor = blue!10]
\vspace{+0.1cm}
\defn Suppose for each $j\in\N$, the operator $D^{(j)}f(x_0)$ exists. The series
$$
\sum_{j=0}^\infty \frac{1}{j!} D^{(j)}f(x_0)\underbrace{(x-x_0,\dots,x-x_0)}_{j\mbox{ times}}
$$
is the \textit{Taylor series} for $f$ centered at $x_0$. If the series converges in some open neighborhood of $x_0$, we say $f$ is \textit{analytic} at $x_0$.
\end{mdframed}
\ex Observe that the function $f:\R\to \R$ defined as
$$
f(x)=\begin{cases}
e^{-1/x},&x>0\\
0,&x\leq 0
\end{cases}
$$
is smooth but not analytic at 0.

\end{document}