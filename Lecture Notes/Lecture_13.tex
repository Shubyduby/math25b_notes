\documentclass[11pt]{article}
\usepackage[utf8]{inputenc}	% Para caracteres en español
\usepackage{amsmath,amsthm,amsfonts,amssymb,amscd}
\usepackage{multirow,booktabs}
\usepackage[table]{xcolor}
\usepackage{fullpage}
\usepackage{lastpage}
\usepackage{enumitem}
\usepackage{fancyhdr}
\usepackage{mathrsfs}
\usepackage{wrapfig}
\usepackage{setspace}
\usepackage{calc}
\usepackage{multicol}
\usepackage{cancel}
\usepackage[retainorgcmds]{IEEEtrantools}
\usepackage[margin=3cm]{geometry}
\usepackage{amsmath}
\newlength{\tabcont}
\setlength{\parindent}{0.0in}
\setlength{\parskip}{0.05in}
\usepackage{empheq}
\usepackage{framed}
\usepackage[most]{tcolorbox}
\usepackage{xcolor}
\colorlet{shadecolor}{orange!15}
\parindent 0in
\parskip 12pt
\geometry{margin=1in, headsep=0.25in}
\theoremstyle{definition}
\newtheorem{defn}{Definition}
\newtheorem{reg}{Rule}
\newtheorem{exer}{Exercise}
\newtheorem{note}{Note}
\newtheorem{prop}{Proposition}
\newtheorem{ex}{Example}
\newcommand{\R}{\mathbb{R}}                      % Wes's shortcut command for the set of all real numbers
\newcommand{\C}{\mathbb{C}}                      % Wes's shortcut command for the set of all complex numbers
\newcommand{\Q}{\mathbb{Q}}
\newcommand{\N}{\mathbb{N}}
\usepackage{enumerate}
\renewcommand\thesection{\S\arabic{section}}
\newtheorem{theorem}{Theorem}
\newtheorem{lem}{Lemma}
\usepackage{mdframed}
\newcommand{\bslash}{\symbol{92}}
\newcommand{\upint}[2]{\overline{\int_{#1}^{#2}}}
\newcommand{\loint}[2]{\underline{\int_{#1}}^{#2}}
\begin{document}


\thispagestyle{empty}

\begin{center}
{\LARGE \bf Lecture 13: Riemann Integrals}\\
{\large Wednesday, 22 February 2023}\\

\end{center}
\section{Partitions}
Now that we have defined the derivative, we turn our attention towards integration. We will begin with Riemann integration for functions of single variables.

\begin{mdframed}[backgroundcolor = blue!10]
\vspace{+0.2cm}
\defn If $I=[a,b]$ is a closed and bounded interval with $a<b$, a \textit{partition} of $I$ is a finite list of points $x_0,x_1,\dots,x_n$ in $I$ such that $a=x_0<x_1<\cdots<x_n=b$.
\end{mdframed}

Using this partition $P$, we can divide the interval $I$ into the subintervals $[x_0,x_1],\dots,[x_{n-1},x_n]$. For each $k=1,\dots,n$, we define $m_k=\inf\{f(x):x\in[x_{k-1},x_k]\}$ and $M_k=\sup\{f(x):x\in[x_{k-1},x_k]\}$. Using this we have the following definitions:

\begin{mdframed}[backgroundcolor = blue!10]
\vspace{+0.2cm}
\defn Let $P=\{x_0,\dots,x_n\}$. The \textit{upper and lower Riemann sums} corresponding to $P$ are defined as $U(f,P)=\sum_{k=1}^n M_k(x_k-x_{k-1})$ and $L(f,P)=\sum_{k=1}^n m_k(x_k-x_{k-1})$, respectively
\end{mdframed}

\note From this, we have that $L(f,P)\leq U(f,P)$ for bounded $f$ and any $P$. Moving forwards, another definition!

\begin{mdframed}[backgroundcolor = blue!10]
\vspace{+0.2cm}
\defn A partition $Q$ is called a \textit{refinement} of $P$ if $P\subset Q$.
\end{mdframed}
\note Notice some consequences of this definition:
\begin{enumerate}
    \item If $Q$ is a refinement of $P$, we hav that $L(f,P)\leq L(f,Q)$ \textit{and} $U(f,P)\geq U(f,Q)$.
    \item If $M=\sup\{f(x):x\in[a,b]\}$, then $-M(b-a)\leq L(f,P)\leq U(f,P)\leq M(b-a)$.
    \item For any arbitrary partition $P$ and a refinement $Q$, $L(f,P)\leq U(f,Q)$.
\end{enumerate}

\section{Integration}
Moving on to another definition that will get us closer to our standard notion of integration,
\begin{mdframed}[backgroundcolor = blue!10]
\vspace{+0.2cm}
\defn Let $f:[a,b]\to \R$ be bounded and $\mathcal{P}$ be the set of all paritions of $[a,b]$. The \textit{lower integral} is defined as the following:
$$
\underline{\int_a}^b f(x)dx = \sup_{p\in\mathcal{P}}\{L(f,P)\}.
$$
The \textit{upper integral} is defined analogously:
$$
\overline{\int_a^b}f(x)dx=\inf_{P\in\mathcal{P}}\{U(f,P)\}.
$$
\end{mdframed}
Here is a related proposition that we will not prove:
\prop If $f:[a,b]\to\R$ is bounded, then
$$
\underline{\int_a}^b f(x)dx\leq\overline{\int_a^b}f(x)dx.
$$
One final definition for today.
\begin{mdframed}[backgroundcolor = blue!10]
\vspace{+0.2cm}
\defn Let $f:[a,b]\to\R$ be bounded. We say that $f$ is \textit{Riemann Integrable} on $[a,b]$ if
$$
\loint{a}{b} f(x)dx=\upint{a}{b} f(x)dx.
$$
\end{mdframed}
\end{document}