\documentclass[11pt]{article}
\usepackage[utf8]{inputenc}	% Para caracteres en español
\usepackage{amsmath,amsthm,amsfonts,amssymb,amscd}
\usepackage{multirow,booktabs}
\usepackage[table]{xcolor}
\usepackage{fullpage}
\usepackage{lastpage}
\usepackage{enumitem}
\usepackage{fancyhdr}
\usepackage{mathrsfs}
\usepackage{wrapfig}
\usepackage{setspace}
\usepackage{calc}
\usepackage{multicol}
\usepackage{cancel}
\usepackage[retainorgcmds]{IEEEtrantools}
\usepackage[margin=3cm]{geometry}
\usepackage{amsmath}
\newlength{\tabcont}
\setlength{\parindent}{0.0in}
\setlength{\parskip}{0.05in}
\usepackage{empheq}
\usepackage{framed}
\usepackage[most]{tcolorbox}
\usepackage{xcolor}
\colorlet{shadecolor}{orange!15}
\parindent 0in
\parskip 12pt
\geometry{margin=1in, headsep=0.25in}
\theoremstyle{definition}
\newtheorem{defn}{Definition}
\newtheorem{reg}{Rule}
\newtheorem{exer}{Exercise}
\newtheorem{note}{Note}
\newtheorem{prop}{Proposition}
\newtheorem{ex}{Example}
\newcommand{\R}{\mathbb{R}}                      % Wes's shortcut command for the set of all real numbers
\newcommand{\C}{\mathbb{C}}                      % Wes's shortcut command for the set of all complex numbers
\newcommand{\Q}{\mathbb{Q}}
\newcommand{\N}{\mathbb{N}}
\usepackage{enumerate}
\renewcommand\thesection{\S\arabic{section}}
\newtheorem{theorem}{Theorem}
\newtheorem{lem}{Lemma}
\usepackage{mdframed}
\newcommand{\bslash}{\symbol{92}}
\newcommand{\upint}[1][2]{\overline{\int_{#1}^{#2}}}
\newcommand{\loint}[1][2]{\underline{\int_{#1}}^{#2}}
\usepackage{dirtytalk}
\newcommand{\bconditions}{\left\{\begin{aligned}}
\newcommand{\econditions}{\end{aligned}\right.}
\newcommand{\mat}{\begin{bmatrix}}
\newcommand{\trix}{\end{bmatrix}}
\newcommand{\dell}{\partial}
\begin{document}


\thispagestyle{empty}

\begin{center}
{\LARGE \bf Lecture 21: Multivariate Calculus}\\
{\large Monday, 20 March 2023}\\
\end{center}
\section{Two Questions Regarding Last Lecture}
 Recall that if $f:\Omega\subseteq \R^n\to\R^m$,
 \begin{enumerate}
     \item If $m=n=1$, differentiability at $x_0$ implies continuity at $x_0$.
     \item If $f$ is differentiable on $\Omega$, then the partial derivatives $\partial f_i/\partial x_j$ exist and the Jacobian is standard matrix for $Df(x)$.
 \end{enumerate}
We have two leading questions. One, is 1 still valid for general $m,n\in\N$?. Two, given that existence of partial derivatives alone is not sufficient to guarantee differentiability, can we formulate stronger conditions that do guarantee it?

\prop The answer to the first question is "yes."

\begin{proof}
    Let $f:\Omega\subseteq\R^n\to\R^m$ where $\Omega$ is open be differentiable. By the differentiability of $f$, given $\varepsilon_0>0$, there exists $\delta_0=\delta_0(\varepsilon_0,x_0)>0$ such that if $x\in\Omega$ and $\|x-x_0\|<\delta_0$, then 
    $$
    \|\underbrace{f(x)-f(x_0)}_y-\underbrace{Df(x_0)(x-x_0)}_z\|\leq \varepsilon_0\|x-x_0\|.
    $$
    Note that $\|y-z\|\geq |\|y\|-\|z\||$. Then we have
    \begin{align*}
        \|f(x)-f(x_0)\|-\|Df(x_0)(x-x_0)\|&\leq\|f(x)-f(x_0)-Df(x_0)(x-x_0)\|\\
        &\leq \varepsilon_0\|x-x_0\|,
    \end{align*}
    and thus
    \begin{align*}
        \|f(x)-f(x_0)\|&\leq \varepsilon_0\|x-x_0\|+\|Df(x_0)(x-x_0)\|\\
        &\leq (\varepsilon_0+\|Df(x_0)\|)\|x-x_0\|.
    \end{align*}
    Given any $\varepsilon>0$, choose $\delta=\min\{\delta_0,\varepsilon/(\varepsilon_0+\|Df(x_0)\|)\}$. Then if $\|x-x_0\|<\delta$ and $x\in\Omega$, then $\|f(x)-f(x_0)\|<\varepsilon$, so $f$ is continuous at $x_0$.
\end{proof}
\ex Consider the function $f:\R\to\R$ defined as
$$
f(x)\left\{\begin{aligned}
&x^2\sin{\frac{1}{x^2}},&x\neq0\\
&0,&x=0.
\end{aligned}\right.
$$
Notice that this function is differentiable everywhere bu $f'$ is unbounded in every open interval containing 0.

We have some special cases for differentiability: when $m=1$ or $n=1$.

\section{Curves in Euclidean Space}

\begin{mdframed}[backgroundcolor = blue!10]
\vspace{+0.2cm}
\defn A \textit{curve} in $\R^n$ is a continuous function $\gamma:I\to\R^n$ where $I$ is an interval in $\R$.

\defn A curve $\gamma:[a,b]\to\R^n$ is \textit{closed} if $\gamma(a)=\gamma(b)$. A curve is \textit{simple} if $\gamma$ is injective on $[a,b]$ except possibly $\gamma(a)=\gamma(b)$.

\defn If $n=2$, a simple, closed curve is called a \textit{Jordan curve}.
\end{mdframed}

\ex The curve $\gamma:[0,4\pi]\to\R^2$ defined as
$$
\gamma(t)=\begin{bmatrix} \cos t\\\sin t\end{bmatrix}.
$$
This curve is closed but not simple.

Now suppose $\gamma:[a,b]\to \R^n$ is differentiable. If $x\in(a,b)$ then $Df(x)$ has  standard matrix
$$
Df(x)=\begin{bmatrix}
    \frac{d\gamma_1}{dx}\\\vdots\\\frac{d\gamma_n}{dx}
\end{bmatrix}.
$$
Then if $\gamma(t)$ is regarded as a "position" at time $t$, then $\gamma'(t)$ can be interpreted as a "velocity" vector at time $t$.

\ex Consider the curve $\gamma:\R\to\R^3$ defined as
$$
\gamma=\begin{bmatrix}
    \cos t\\ \sin t\\ t
\end{bmatrix}.
$$
This is a helix. Wes eats a piece of chalk at this point. We then have that
$$
\gamma'(t)=\begin{bmatrix}
    -\sin t\\ \cos t\\ 1
\end{bmatrix}\implies
\gamma'(2\pi)=\begin{bmatrix}
    0\\1\\1\\
\end{bmatrix}.
$$
A parameterization for the lines tangent to $t=2\pi$ and parallel to the velocity vector is
$$
L(s)=\begin{bmatrix}
    1\\ 0 \\ 2\pi
\end{bmatrix}+s\begin{bmatrix}
    0\\1\\1
\end{bmatrix}
$$
\note There are infinitely many lines tangent to the curve at any point!

\ex Consider the curve $\gamma:\R\to \R^2$ defined as
$$
\gamma(t)=\mat t^3\\t^2\trix\implies \gamma'(t)=\mat 3t^2\\2t\trix.
$$
Then $\gamma(t)$ is differentiable. But notice the curve is not \textit{smooth} so it is not \textit{regular}. This \textit{can} happen for curves such that $\gamma'(t_0)=\begin{bmatrix}
    0\\\vdots\\0
\end{bmatrix}.$

\ex Consider the curve
$$
\gamma(t)=\mat t^2\\t^2\trix, t\in[-1,1].
$$
goofy ah curve.

\prop Let $f:\Omega\subseteq\R^n\to\R^m$ where $\Omega$ is open. If all partial derivatives $\partial f_i/\partial x_j$ exist and are continuous in $\Omega$ then $f$ is differentiable in $\Omega$. 
\begin{proof}
    The full proof is in Marsden and Hoffman. Here is a sketch of the proof.

    Let $x\in\Omega$. We want to show that given some $\varepsilon>0$, there exists a $\delta>0$ such that if we pick a point $y\in\Omega$ with $\|x-y\|<\delta$, then there exists a linear transformation $Df(x)$ such that
    $$
    \|f(y)-f(x)-Df(x)(y-x)\|\leq \varepsilon\|x-y\|.
    $$
    Here are some simplifications that make the proof more manageable:
    \begin{enumerate}
        \item It suffices to assume that $m=1$, that is, this applies when $f:\R\to\R$ works. This is because we can use the fact that the Euclidean norm in $\R^m$ satisfies
        $$
        \|z\|\leq |z_1|+|z_2|+\cdots+|z_m|
        $$
        for all $z\in \R^m$. So we can just work component-wise and then compare the $m$-dimensional norm to the sum of all the 1-dimensional norm.

        \item We know that if $Df(x)$ is to exist, then it must have the Jacobian as its standard matrix. Let that guide the proof!

        \item If $f:\Omega\subseteq\R^n\to\R$, write 
        \begin{align*}
            f(y)-f(x)&=f(y_1,y_2,\dots,y_n)-f(x_1,x_2,\dots,x_n)\\
            &=f(y_1,y_2,\dots,y_n)-f(x_1,y_2,\dots,y_n)\\
            &+f(x_1,y_2,\dots.y_n)-f(x_1,x_2,y_3,\dots,y_n)\\
            &\vdots\\
            &+f(x_1,x_2,\dots,x_{n-1},y_n)-f(x_1,x_2,\dots,x_n).
        \end{align*}
        and apply the mean value theorem to each of these lines (for example, for the first line of the long expression we can say it equals $\frac{\dell f}{\dell x_1} (c_1)(y_1-x_1)$ for some $c_1\in \R$). This is like moving component-wise and then aggregating everything to get a whole picture of the multivariate function.
    \end{enumerate}
\end{proof}
Here is a useful generalization of partial derivatives:
\begin{mdframed}[backgroundcolor = blue!10]
\vspace{+0.2cm}
\defn Suppose $f:\Omega\subseteq \R^n\to \R$ where $\Omega$ is open and $x_0\in \Omega$. Given a unit vector $u\in\R^n$, the \textit{directional derivative} of $f$ at $x_0$ is 
$$
\lim_{h\to0} \frac{f(x_0+hu)-f(x_0)}{h}.
$$
This is notated as $D_uf(x_0)$.
\end{mdframed}
\end{document}
