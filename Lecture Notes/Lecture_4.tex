\documentclass[11pt]{article}
\usepackage[utf8]{inputenc}	% Para caracteres en español
\usepackage{amsmath,amsthm,amsfonts,amssymb,amscd}
\usepackage{multirow,booktabs}
\usepackage[table]{xcolor}
\usepackage{fullpage}
\usepackage{lastpage}
\usepackage{enumitem}
\usepackage{fancyhdr}
\usepackage{mathrsfs}
\usepackage{wrapfig}
\usepackage{setspace}
\usepackage{calc}
\usepackage{multicol}
\usepackage{cancel}
\usepackage[retainorgcmds]{IEEEtrantools}
\usepackage[margin=3cm]{geometry}
\usepackage{amsmath}
\newlength{\tabcont}
\setlength{\parindent}{0.0in}
\setlength{\parskip}{0.05in}
\usepackage{empheq}
\usepackage{framed}
\usepackage[most]{tcolorbox}
\usepackage{xcolor}
\colorlet{shadecolor}{orange!15}
\parindent 0in
\parskip 12pt
\geometry{margin=1in, headsep=0.25in}
\theoremstyle{definition}
\newtheorem{defn}{Definition}
\newtheorem{reg}{Rule}
\newtheorem{exer}{Exercise}
\newtheorem{note}{Note}
\newtheorem{prop}{Proposition}
\newtheorem{ex}{Example}
\newcommand{\R}{\mathbb{R}}                      % Wes's shortcut command for the set of all real numbers
\newcommand{\C}{\mathbb{C}}                      % Wes's shortcut command for the set of all complex numbers
\newcommand{\Q}{\mathbb{Q}}
\newcommand{\N}{\mathbb{N}}
\usepackage{enumerate}
\renewcommand\thesection{\S\arabic{section}}
\newtheorem{theorem}{Theorem}
\newtheorem{lem}{Lemma}
\usepackage{mdframed}
\newcommand{\bslash}{\symbol{92}}

\begin{document}


\thispagestyle{empty}

\begin{center}
{\LARGE \bf Lecture 4: Limit Points and Closures}\\
{\large Monday, 30 January 2023}\\

\end{center}

\section{Limit Points}
Recall from the problem set the definition of a limit point:
\defn Let $(M,d)$ be metric space and $E\subseteq M$. We say that $p\in M$ is a \textit{limit point} of $E$ if every open neighborhood of $p$ contains some $q\in E$ with $q\neq p$.

Limit points would them seem to have a close relation to sequences. Let us solidify this relationship:

\prop Let $(M,d)$ be metric space and $E\subseteq M$. A point $p\in M$ is a limit point of $E$ if and only if $\{x_n\}\to p$ as $n\to\infty$ for some sequence $\{x_n\}\in E\bslash\{p\}$.

\begin{proof}
    Suppose that $p$ is a limit point. Then consider the open ball $B(p,1)$. Since $p$ is a limit point of $E$, there exists a point $x_1\in B(p,1)\cap E$ such that $x_1\neq p$. Like wise, we can pick any $x_n\in B(p,1/n)\cap E$ such that $x_n\neq p$. This creates a series that converges to $p$.  
\end{proof}
The converse of this proof is in the textbook.

Recall that for any $E\subseteq M$, the interior of $E$, notated int$(E)$, the largest open subset of $E$. 

\defn The \textit{closure} of $E$ is the intersection of all closed sets containing $E$.

\note This is equivalent to saying that the closure of $E$ is smallest closed set containing $E$.

\ex If $E=[0,1)$, then int$(E)=(0,1)$ and cl$(E)=[0,1]$.
\end{document}