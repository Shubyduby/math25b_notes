\documentclass[11pt]{article}
\usepackage[utf8]{inputenc}	% Para caracteres en español
\usepackage{amsmath,amsthm,amsfonts,amssymb,amscd}
\usepackage{multirow,booktabs}
\usepackage[table]{xcolor}
\usepackage{fullpage}
\usepackage{lastpage}
\usepackage{enumitem}
\usepackage{fancyhdr}
\usepackage{mathrsfs}
\usepackage{wrapfig}
\usepackage{setspace}
\usepackage{calc}
\usepackage{stmaryrd}
\usepackage{multicol}
\usepackage{cancel}
\usepackage[retainorgcmds]{IEEEtrantools}
\usepackage[margin=3cm]{geometry}
\usepackage{amsmath}
\newlength{\tabcont}
\setlength{\parindent}{0.0in}
\setlength{\parskip}{0.05in}
\usepackage{empheq}
\usepackage{framed}
\usepackage[most]{tcolorbox}
\usepackage{xcolor}
\colorlet{shadecolor}{orange!15}
\parindent 0in
\parskip 12pt
\geometry{margin=1in, headsep=0.25in}
\theoremstyle{definition}
\newtheorem{defn}{Definition}
\newtheorem{reg}{Rule}
\newtheorem{exer}{Exercise}
\newtheorem{note}{Note}
\newtheorem{prop}{Proposition}
\newtheorem{ex}{Example}
\newcommand{\R}{\mathbb{R}}                      % Wes's shortcut command for the set of all real numbers
\newcommand{\C}{\mathbb{C}}                      % Wes's shortcut command for the set of all complex numbers
\newcommand{\Q}{\mathbb{Q}}
\newcommand{\N}{\mathbb{N}}
\usepackage{enumerate}
\renewcommand\thesection{\S\arabic{section}}
\newtheorem{theorem}{Theorem}
\newtheorem{lem}{Lemma}
\usepackage{mdframed}
\newcommand{\bslash}{\symbol{92}}
\newcommand{\upint}[1][2]{\overline{\int_{#1}^{#2}}}
\newcommand{\loint}[1][2]{\underline{\int_{#1}}^{#2}}
\usepackage{dirtytalk}
\newcommand{\bconditions}{\left\{\begin{aligned}}
\newcommand{\econditions}{\end{aligned}\right.}
\newcommand{\mat}{\begin{bmatrix}}
\newcommand{\trix}{\end{bmatrix}}

\newcommand{\dell}{\partial}
\begin{document}


\thispagestyle{empty}

\begin{center}
{\LARGE \bf Extra Video: The Divergence Theorem and Stokes' Theorem}\\
{\large Monday, 19 April 2023}\\
\end{center}
\section{Manifolds}
Before we get into the presentation of the divergence theorem and Stokes' theorem, we equipt ourselves with some definitions from Spivak's \textit{Calculus on Manifolds}. His definition of manifold is only slightly different form ours but we will use his definition for consistency within this document. 

Also, note that in this document, \textit{diffeomorphism} will mean $C^\infty$ diffeomorphism since Spivak uses \say{differentiable} to mean \say{infinitely differenitable.}  
\begin{mdframed}[backgroundcolor = blue!10]
\vspace{+0.1cm}
\defn A subset $M\subseteq \R^n$ is called a \textit{k-dimensional manifold} (in $\R^n$) if for every point $x\in M$ there is an open set $U$ containing $x$, an open set $V\subseteq \R^n$, and a diffeomorphism $h:U\to V$ such that
$$
h(U\cap M) = V\cap (\R^k \times \{0\})=\{y\in V: y^{k+1}=\cdots = y^n = 0\}.
$$
\end{mdframed}
In other words, $U\cap M$ is, \say{up to diffeomorphism,} simply $\R^k\times \{0\}$.

\begin{mdframed}[backgroundcolor = blue!10]
\vspace{+0.1cm}
\defn The \textit{half-space} $H^k\subseteq \R^k$ is defined as $\{x\in \R^k : x_k\geq 0\}$. 

\defn A subset $M\subseteq \R^n$ is a \textit{k-dimensional manifold with boundary} if for every point $x\in M$ either the condition in Definition 1 is satisfied or there exists an open set $U$ containing $x$, an open set $V\subseteq \R^n$, and a diffeomorphism $h:U\to V$ such that
$$
h(U\cap M) = V\cap (H^k\times \{0\}) = \{y\in V: y^k\geq 0\mbox{ and }y^{k+1}=\cdots=y^n=0\}
$$
and $h(x)$ has a $k$th component equal to 0.
\end{mdframed}
\section{The Divergence Theorem}
We now present the divergence theorem in three dimensions:
\begin{shaded}
\theorem (\textit{Divergence Theorem})(\textit{Three Dimensions}) Let $M\subseteq \R^3$ be a compact, connected smooth $3$-manifold with boundary and let $N$ denote the unit outward normal on $\dell M$. If $F$ is a $C^1$ vector field on some open set containing $M$, then
$$
\int_M \mathrm{div}F = \int_{\dell M} \langle F, N\rangle.
$$
\end{shaded}
\ex Let $M\subseteq \R^3$ be defined by the inequalities $x^2+y^2\leq z\leq 4$ and 
$$
F\left(\begin{bmatrix} x \\ y \\ z\end{bmatrix}\right)=\mat x^2 \\ -z \\ y \trix.
$$
Evaluate
$$
\int_{\dell M} F
$$
with $\dell M$ oriented with outward normals.

\textit{Solution:} By Theorem 1, we have that
$$
\int_{\dell M} F = \int_{\dell M} \langle F,N\rangle =\int_M \mathrm{div} F = \iiint_M 2x.
$$
Using cylindrical coordinates, we can rewrite out integral as
$$
\iiint_M 2x = \int_0^{2\pi} \int_0^2 \int_{r^2}^4 (2r\cos\theta)r~dz\,dr\,d\theta =0.
$$

\section{Stokes' Theorem}
\subsection{Curl}
Recall the statement of Green's theorem, in particular, the equality
$$
\oint_C F = \iint_D \frac{\dell F_2}{\dell x}-\frac{\dell F_1}{\dell y},
$$
where $F:\R^2\to \R^2$. The term
$$
\frac{\dell F_2}{\dell x}-\frac{\dell F_1}{\dell y}
$$
is a representation of \textit{two-dimensional curl} of the vector field. It can be interpreted as the amount of clockwise spin a pinwheel would have under the influence of the vector field at a given point. We want to have a notation of this in three dimensions as well.
\begin{mdframed}[backgroundcolor = blue!10]
\vspace{+0.1cm}
\defn If $F:\R^3\to\R^3$ is of class $C^1$, the \textit{curl} of $F$ at $(x_0,y_0,z_0)$ is the vector
$$
\mathrm{curl}\,F(x_0,y_0,z_0) = \mat (F_3)_y - (F_2)_z\\ (F_1)_z - (F_3)_x \\ (F_2)_x - (F_1)_y \trix
$$
evaluated at $(x_0,y_0,z_0)$.
\end{mdframed}
\note The curl will sometimes be written as 
$$
\nabla \times F.
$$
This is entirely notational, and you can think of this similarly to how we expressed divergence as $\nabla \times F$. In particular, we let $\nabla = \mat \frac{\dell}{\dell x}&\frac{\dell}{\dell y}&\frac{\dell}{\dell z}\trix$.

We can interpret curl in three dimensions as placing paddler wheel in a fluid with velocity field $F$. Letting the unit vector $N$ be a fixed axis of rotation of the paddle wheel, the amount of clockwise spin the paddle wheel will have (looking down onto the wheel with the axis of the paddle wheel faced towards you) is calculated as
$$
(\mathrm{curl}\,F)\cdot N.
$$
\subsection{Stokes' Theorem}
This is a rather non-technical statement of the theorem:
\begin{shaded}
    \theorem (\textit{Stokes}) Assume that
    \begin{enumerate}
        \item The subset $S\subseteq \R^3$ is a compact and connected 2-manifold that admits a $C^1$ regular parameterization.
        \item The \say{boundary} of $S$ is a closed curve $C$ that has a piecewise $C^1$ regular parameterization. Note that now, when we say boundary, we do not mean it in the way we defined it earlier in the class. Now, we mean it as an edge where if you cross the edge, you reach the other side of the surface.
        \item $S$ is oriented and the orientation of $S$ induces an orientation of $C$. Here is how the orientation is induced: imagine a little guy walking on the boundary of $S$ with their head pointing in the direction of the normal vector used to orient the surface. As the person walks along $C$, the orientation of $C$ must make sure that the surface $S$ is always to the left of the guy.
        \item The vector field $F:\R^3\to \R^3$ is of class $C^1$.

        Under the assumptions above, 
        $$
        \oint_C F = \int_S \mathrm{curl}\, F.
        $$
    \end{enumerate}
\end{shaded}
\note This is a generalization of Green's theorem.
\end{document}

