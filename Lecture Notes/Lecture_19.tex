\documentclass[11pt]{article}
\usepackage[utf8]{inputenc}	% Para caracteres en español
\usepackage{amsmath,amsthm,amsfonts,amssymb,amscd}
\usepackage{multirow,booktabs}
\usepackage[table]{xcolor}
\usepackage{fullpage}
\usepackage{lastpage}
\usepackage{enumitem}
\usepackage{fancyhdr}
\usepackage{mathrsfs}
\usepackage{wrapfig}
\usepackage{setspace}
\usepackage{calc}
\usepackage{multicol}
\usepackage{cancel}
\usepackage[retainorgcmds]{IEEEtrantools}
\usepackage[margin=3cm]{geometry}
\usepackage{amsmath}
\newlength{\tabcont}
\setlength{\parindent}{0.0in}
\setlength{\parskip}{0.05in}
\usepackage{empheq}
\usepackage{framed}
\usepackage[most]{tcolorbox}
\usepackage{xcolor}
\colorlet{shadecolor}{orange!15}
\parindent 0in
\parskip 12pt
\geometry{margin=1in, headsep=0.25in}
\theoremstyle{definition}
\newtheorem{defn}{Definition}
\newtheorem{reg}{Rule}
\newtheorem{exer}{Exercise}
\newtheorem{note}{Note}
\newtheorem{prop}{Proposition}
\newtheorem{ex}{Example}
\newcommand{\R}{\mathbb{R}}                      % Wes's shortcut command for the set of all real numbers
\newcommand{\C}{\mathbb{C}}                      % Wes's shortcut command for the set of all complex numbers
\newcommand{\bconditions}{\left\{\begin{aligned}}
\newcommand{\econditions}{\end{aligned}\right.}
\newcommand{\Q}{\mathbb{Q}}
\newcommand{\N}{\mathbb{N}}
\usepackage{enumerate}
\renewcommand\thesection{\S\arabic{section}}
\newtheorem{theorem}{Theorem}
\newtheorem{lem}{Lemma}
\usepackage{mdframed}
\newcommand{\bslash}{\symbol{92}}
\newcommand{\upint}[1][2]{\overline{\int_{#1}^{#2}}}
\newcommand{\loint}[1][2]{\underline{\int_{#1}}^{#2}}
\newcommand{\mat}{\begin{bmatrix}}
\newcommand{\trix}{\end{bmatrix}}
\begin{document}


\thispagestyle{empty}

\begin{center}
{\LARGE \bf Lecture 19: Initial Value Problems}\\
{\large Wednesday, 8 March 2023}\\
\end{center}

\section{Existence and Uniqueness}
Last lecture we found that the initial value problem
$$
\left\{\begin{aligned}
&\frac{dx}{dt}=3x^\frac{2}{3}\\
&x(0)=0
\end{aligned}\right.
$$
has many solutions. Again, this is a problem in mathematical modeling when we would like to have just one solution to make a model or predictions. Then we ask the question: given an initial value problem
$$
*\left\{\begin{aligned}
    &\frac{dx}{dt}=f(x)\\
    &x(0)=b
\end{aligned}\right.,
$$
what conditions on $f$ guarantee the existence of a unique solution, at least for some interval $(\alpha,\beta)$ where $\alpha<0<\beta$?

\prop With the notation above, suppose $f:\Omega\subseteq \R\to \R$ is continuous on the open set $\Omega$ and $b\in\Omega$. Then $x\in C^1((\alpha,\beta),\Omega)$ satisfies $*$ if and only if $x$ satisfies
$$
x(t)=b+\int_0^t f(x(s))ds
$$
for all $t\in(\alpha,\beta)$. 

\begin{proof}
Apply the fundamental theorem of calculus.
\end{proof}

\begin{shaded}
\theorem (\textit{Picard-Lindelöf}) Suppose $f:\Omega\subseteq \R\to\R$ is locally Lipschitz on $\Omega$, where $\Omega$ is open, and $b\in \Omega$. Then there exists $\eta>0$ and a unique, differentiable function $x:(-\eta,\eta)\to\Omega$ such that
$$
*\bconditions
&\frac{dx}{dt}=f(x)\\
&x(0)=b
\econditions.
$$
\end{shaded}
The general plan for proving this will be to rewrite $*$ in terms of an integral and setting up a fixed point result using contraction mapping.
\begin{proof}
    Suppose the assumptions of the theorem are met. Then since $\Omega$ is an open set, $b\in\Omega$, and $f$ is locally Lipshitz over $\Omega$, we can choose a $\delta>0$ such that the set $[b-\delta,b+\delta]\subseteq \Omega$ and the restriction $f|_{[b-\delta,b+\delta]}$ is Lipschitz with constant $L$. With the $\delta$ defined as above, let us define the set of functions $S_\eta\subseteq C([-\eta,\eta],\R)$ as $S_\eta=\{x\in C([-\eta,\eta],\R):|x(t)-b|<\delta\;\forall t\in [-\eta,\eta]\}$. 

    Now consider any function $x\in S_\eta$. By definition, $x$ is a continuous function and $x(t)\in [b-\delta,b+\delta]\subseteq \Omega$ for all $t\in [-\eta,\eta]$. We now define a mapping $\Phi: S_\eta\to C([-\eta,\eta],\R)$ as 
    $$
    \Phi(x)(t)=b+\int_0^t f(x(s))~ds
    $$
    for all $t\in[-\eta,\eta]$. This mapping is well defined since $f$ is locally Lipschitz and thus continuous so the function $f\circ x$ is continuous. Notice that finding a fixed point of $\Phi$ is equivalent to finding a solution of $*$ by Proposition 1.

    We now will show that if $\eta$ is suitably small, then $\Phi:S_\eta\to S_\eta$. Note that if $x\in S_\eta$, then
    $$
    \|\Phi(x)-b\|=\sup_{t\in[-\eta,\eta]}|\Phi(x)(t)-b|=\sup_{t\in[-\eta,\eta]} \left|\int_0^t f(x(s))~ds\right|.
    $$
    Let $I(t)=[\min\{0,t\},\max\{0,t\}]$. We have that
    $$
    \|\Phi(x)-b\|\leq \sup_{t\in[-\eta,\eta]} \int_{I(t)} |f(x(s))|~ds.
    $$
    Since $I(t)$ is compact and $|f\circ x|$ is a continuous function, we know that by the extreme value theorem, we can choose a constant $k$ such that 
    $$
    \max_{z\in[b-\delta,b+\delta]}|f(z)|=k.
    $$
    Then 
    $$
    \sup_{t\in[-\eta,\eta]}\int_{I(t)}|f(x(s))|~ds\leq \sup_{t\in[-\eta,\eta]}|t|k=\eta k.
    $$
    Pick $x,y\in S_\eta$. Observe that
    $$
    \begin{aligned}
        \|\Phi(x)-\Phi(y)\|&=\sup_{t\in[-\eta,\eta]}|\Phi(x)(t)-\Phi(y)(t)|\\
        &= \sup_{t\in[-\eta,\eta]} \left|\int_0^t f(x(s))~ds-\int_0^t f(y(s))~ds\right|\\
        &\leq \sup_{t\in[-\eta,\eta]}\int_{I(t)} |f(x(s))-f(y(s))|~ds\\
        &\leq \sup_{t\in[-\eta,\eta]}\int_{I(t)} L|x(s)-y(s)|~ds\\
        &\leq \sup_{t\in[-\eta,\eta]}\int_{I(t)} L\|x-y\|~ds\\
        &=\sup_{t\in[-\eta,\eta]} |t|L\|x-y\|\\
        &=\eta L\|x-y\|.
    \end{aligned}
    $$
    Then we can choose an $\eta>0$ small enough such that $\eta L<1$, making $\Phi$ a contraction, and $\eta k\leq \delta$, so $\Phi:S_\eta\to S_\eta$. Since $S_\eta$ is a closed subset of the Banach space $C([-\eta,\eta],\R)$, the contraction mapping theorem tells us that there exists a unique function $x\in S_\eta$ such that $\Phi(x)=x$. Then by Proposition 1, the unique function $x$ satisfies the initial value problem $*$, proving existence and uniqueness of the solution.
\end{proof}
\end{document}