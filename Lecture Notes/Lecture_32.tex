\documentclass[11pt]{article}
\usepackage[utf8]{inputenc}	% Para caracteres en español
\usepackage{amsmath,amsthm,amsfonts,amssymb,amscd}
\usepackage{multirow,booktabs}
\usepackage[table]{xcolor}
\usepackage{fullpage}
\usepackage{lastpage}
\usepackage{enumitem}
\usepackage{fancyhdr}
\usepackage{mathrsfs}
\usepackage{wrapfig}
\usepackage{setspace}
\usepackage{calc}
\usepackage{stmaryrd}
\usepackage{multicol}
\usepackage{cancel}
\usepackage[retainorgcmds]{IEEEtrantools}
\usepackage[margin=3cm]{geometry}
\usepackage{amsmath}
\newlength{\tabcont}
\setlength{\parindent}{0.0in}
\setlength{\parskip}{0.05in}
\usepackage{empheq}
\usepackage{framed}
\usepackage[most]{tcolorbox}
\usepackage{xcolor}
\colorlet{shadecolor}{orange!15}
\parindent 0in
\parskip 12pt
\geometry{margin=1in, headsep=0.25in}
\theoremstyle{definition}
\newtheorem{defn}{Definition}
\newtheorem{reg}{Rule}
\newtheorem{exer}{Exercise}
\newtheorem{note}{Note}
\newtheorem{prop}{Proposition}
\newtheorem{ex}{Example}
\newcommand{\R}{\mathbb{R}}                      % Wes's shortcut command for the set of all real numbers
\newcommand{\C}{\mathbb{C}}                      % Wes's shortcut command for the set of all complex numbers
\newcommand{\Q}{\mathbb{Q}}
\newcommand{\N}{\mathbb{N}}
\usepackage{enumerate}
\renewcommand\thesection{\S\arabic{section}}
\newtheorem{theorem}{Theorem}
\newtheorem{lem}{Lemma}
\usepackage{mdframed}
\newcommand{\bslash}{\symbol{92}}
\newcommand{\upint}[1][2]{\overline{\int_{#1}^{#2}}}
\newcommand{\loint}[1][2]{\underline{\int_{#1}}^{#2}}
\usepackage{dirtytalk}
\newcommand{\bconditions}{\left\{\begin{aligned}}
\newcommand{\econditions}{\end{aligned}\right.}
\newcommand{\mat}{\begin{bmatrix}}
\newcommand{\trix}{\end{bmatrix}}
\newcommand{\dell}{\partial}
\begin{document}


\thispagestyle{empty}

\begin{center}
{\LARGE \bf Lecture 32: Line Integrals and Associated Theorems}\\
{\large Friday, 14 April 2023}\\
\end{center}
\section{Line Integrals}
\subsection{Scalar-Valued Line Integrals}
We now study the integrals of curves, particular $C^1$ regular curves. If $\gamma:[a,b]\to \R^n$ is $C^1$ on some open set containing $[a,b]$ and $\gamma'(t)\neq 0$ for all $t\in[a,b]$, then we have that
\begin{enumerate}
    \item The length of $C = \gamma([a,b])$ is finite. In particular, 
    $$
    \int_a^b \|\gamma' (t)\|~dt
    $$
    exists because $\|\gamma'\|$ is a continuous function and $[a,b]$ is bounded.
    \item The condition $\gamma'(t)\neq 0$ guards against differentiable curves that fail to be regular.
\end{enumerate}
For curves that satisfy the conditions above, we can define a line integral. These apply to both scalar and vector-valued functions. 

\begin{mdframed}[backgroundcolor = blue!10]
\vspace{+0.1cm}
\defn Let $f:\Omega\subseteq \R^n\to \R$ be continuous on the open set $\Omega$. Let $C$ be a curve contained in $\Omega$ that admits a $C^1$ regular parameteization $\gamma:[a,b]\to \Omega$. The \textit{line integral} $\int_C f~dt$ is defined as 
$$
\int_a^b f(\gamma(t))\|\gamma'(t)\|~dt.
$$
\end{mdframed}
\note This is actually independent of which $C^1$ regular parameterization is used.

A nice interpretation of this: If $f(x)$ represent a density at point $x$, then the line integral $\int_C f~ds$ represents the mass of the 1-dimensional wire formed by the curve.

\subsection{Vector-Valued Line Integrals}
Again, let some curve $C$ be a $C^1$ regular curve in $\R^n$. We now can defined a vector-valued line integral.

\begin{mdframed}[backgroundcolor = blue!10]
\vspace{+0.1cm}
\defn A curve $\gamma:[a,b]\to \Omega$ is \textit{oriented} if it starts at $\gamma(a)$ and ends at $\gamma(b)$.

\defn If $F:\Omega\subseteq \R^n\to \R^m$ is continuous on the open set $\Omega$ and $C$ is an oriented curve contained in $\Omega$ that admits a $C^1$ regular parameterization $\gamma:[a,b]\to \Omega$ then the \textit{line integral} denoted as $\int_C F~dt$ is defined as 
$$
\int_a^b \langle F(\gamma(t)),\gamma'(t)\rangle ~dt.
$$
\end{mdframed}
\note Like the scalar valued line integral, the value of the integral is \textit{independent} of which $C^1$ regular parameterization is being used for the curve.

\ex Consider the function $F:\R^2\to \R^2$ defined by
$$
\mat x\\y\trix \mapsto \mat -y \\ x \trix
$$
and let $C$ be the line segment going from $(0,0)$ to $(-3,4)$. Evaluate
$$
\int_C F~ds.
$$

\textit{Solution:} One parametrization for he line segment $C$ we can use is
$$
\gamma(t)=\mat -3t \\ 4t \trix \implies \gamma'(t) = \mat -3 \\ 4 \trix.
$$
Then we have the composition
$$
F(\gamma(t)) =\mat -4t \\ 3t \trix.
$$
Then we evaluate the integral
$$
\begin{aligned}
\int_C F~ds = \int_0^1 \left\langle \mat -4t\\ -3t \trix, \mat -3 \\ 4 \trix \right\rangle ~dt= \int_0^1 0 ~dt=0.
\end{aligned}
$$

Compare this to if we have evaluated the line integral going along some new curve $\xi$ from $(0,0)$ to $(-3,4)$. In this case, let it be the curve parameterized by 
$$
\eta(t)=\mat 3\cos(t) - 3 \\ 4\sin(t) \trix, \quad t\in [0,\pi/2]. 
$$
Then we have that
$$
\begin{aligned}
\int_\xi F~ds &= \int_0^{\pi/2} \left\langle \mat -4\sin (t) \\ 3\cos(t)-3 \trix, \mat -3\sin t \\ 4\cos t \trix \right\rangle~dt \\
&=\int_0^{2/\pi} 12\sin^2 t +12\cos^2t -12\cos t ~dt \\
&= \left. 12t - 12\sin t \right|_0^{\pi/2}\\
&= 12\left(\frac{\pi}{2}-1\right)\neq 0.
\end{aligned}
$$
\note It is quite common to work with \say{natural} parameterization. For example, we often like ot parameterize by arc length:
$$
r(t) = \int_a^t \|\gamma(\tau)\|~d\tau.
$$

\note If $-C$ is the same curve as some curve $C$ but with opposite orientation, then
$$
\int_{-C} F~ds = -\int_C F~ds.
$$
\note Changing the path $C$ can change $\int_C F~ds$ (e.g. in Example 1). However, this is not always the case, in particular, when the integral is path-independent. 
\section{Properties of Vector Fields}

In the spirit of Note 5, we want to know when integrals are path independent.
\begin{mdframed}[backgroundcolor = blue!10]
\vspace{+0.1cm}

\defn A vector field $F:\Omega\subseteq \R^n\to\R^n$, where $\Omega$ is open, is called \textit{conservative} if there exists a $C^1$ function $f:\Omega\subseteq \R^n\to \R$ such that $F=\nabla f$ on $\Omega$. Such  function $f$ is called a \textit{potential function} for $F$.

\end{mdframed}

\ex Is the function $F:\R^2\to \R^2$ defined as
$$
F\left(\mat x\\y \trix\right) = \mat -y \\ x \trix 
$$
conservative?

\textit{Solution:} If there exists some $C^1$ function $f:\R^2\to \R$ such that $F=\nabla f$, then we have that
$$
\frac{\dell f}{\dell x}=-y,\quad \frac{\dell f}{\dell y}=x.
$$
However, notice that 
$$
\frac{\dell^2 f}{\dell y\dell x}=-1 \quad \frac{\dell^2 f}{\dell x \dell y} = 1,
$$
which cannot happen by Clairaut's theorem. Then there exists no such potential function for $F$ and thus $F$ is not conservative.

\prop If $F:\Omega\subseteq \R^2\to \R^2$ is of class $C^1$ and $F$ is conservative, then
$$
\frac{\dell F_1}{\dell y}=\frac{\dell F_2}{\dell x}.
$$
\begin{proof}
    The proof follows directly from Clairaut's theorem.
\end{proof}

\note The proposition only goes in \textit{one} direction.

\ex Then vector field
$$
F\left(\mat x \\ y \trix\right)=\frac{1}{x^2+y^2}\mat -y \\ x \trix 
$$
defined over $\R^2\bslash\{0\}$ is not conservative even though $\dell F_1/\dell x = \dell F_2/\dell y$.

\prop Let $F:\Omega\subseteq \R^n \to \R^n$ have a $C^1$ potential function $f:\Omega\subseteq\R^n \to \R$. Then given any points $A,B\in \Omega$ and any oriented $C^1$ regular curve in $\Omega$ from $A$ to $B$, we have that
$$
\int_C F~ds = f(B)=f(A).
$$

\begin{proof}
    Let $\gamma:[a,b]\to \Omega$ parameterize the curve $C$, $\gamma(a)=A$, and $\gamma(b)=B$. Then , by the cahin rule, we have that
    $$
    \begin{aligned}
        \int_C F~ds &= \int_a^b \langle F(\gamma(t)),\gamma'(t)\rangle ~dt\\
        &= \int_a^b \langle \nabla f(\gamma(t)),\gamma'(t)\rangle ~dt\\
        &= \int_a^b \frac{d}{dt}[f(\gamma(t))]~dt\\
        &= f(\gamma(b))-f(\gamma(a))\\
        &= f(B)-f(A).
    \end{aligned}
    $$
\end{proof}
\note If $F$ has a potential function on an open set $\Omega$, this result still if holds if $C$ is continuous and only piecewise $C^1$ regular. This is helpful for polygonal paths.

\note If $F$ is conservative, $\int_C F~ds$ has \textit{independence of path}.

\textit{Corollary:} If $F$ is conservative and $C$ is a curve that is $C^1$, regular, closed, and simple, then
$$
\oint_C F~ds=0
$$
because $A=B$ so $f(A)-f(B)=0$.

\section{Towards Green's Theorem}

Recall now that a simple, closed and continuous curve in $\R^2$ is called a Jordan curve. Then if $\gamma:[0,1]\to\R^2$ is a Jordan curve, we have that $\gamma(0)=\gamma(1)$ and $\gamma|_{[0,1]}$ is injective (excluding endpoints).

\begin{shaded}
\theorem (\textit{Jordan Curve Theorem}) If $C$ is a Jordan curve in $\R^2$ then $\R^2\bslash C$ consists of exactly 2 connected components.
\end{shaded}
\begin{proof}
    (\textit{Proof by it's obvious}) It's obvious. Thing inside, thing outside,
\end{proof}

Jokes aside, this is difficult to formally prove.

\begin{mdframed}[backgroundcolor = blue!10]
\vspace{+0.1cm}
\defn Let $C$ be an oriented Jordan curve in $\R^2$ with parameterization $\gamma:[a,b] \to \R^2$. We say that $C$ is \textit{positively oriented} if the interior component of $\R^2\bslash C$  is always left of $\gamma(t)$ as we follow $C$ in the direction of increasing $t$.

\end{mdframed}
\end{document}