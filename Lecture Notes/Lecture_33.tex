\documentclass[11pt]{article}
\usepackage[utf8]{inputenc}	% Para caracteres en español
\usepackage{amsmath,amsthm,amsfonts,amssymb,amscd}
\usepackage{multirow,booktabs}
\usepackage[table]{xcolor}
\usepackage{fullpage}
\usepackage{lastpage}
\usepackage{enumitem}
\usepackage{fancyhdr}
\usepackage{mathrsfs}
\usepackage{wrapfig}
\usepackage{setspace}
\usepackage{calc}
\usepackage{stmaryrd}
\usepackage{multicol}
\usepackage{cancel}
\usepackage[retainorgcmds]{IEEEtrantools}
\usepackage[margin=3cm]{geometry}
\usepackage{amsmath}
\newlength{\tabcont}
\setlength{\parindent}{0.0in}
\setlength{\parskip}{0.05in}
\usepackage{empheq}
\usepackage{framed}
\usepackage[most]{tcolorbox}
\usepackage{xcolor}
\colorlet{shadecolor}{orange!15}
\parindent 0in
\parskip 12pt
\geometry{margin=1in, headsep=0.25in}
\theoremstyle{definition}
\newtheorem{defn}{Definition}
\newtheorem{reg}{Rule}
\newtheorem{exer}{Exercise}
\newtheorem{note}{Note}
\newtheorem{prop}{Proposition}
\newtheorem{ex}{Example}
\newcommand{\R}{\mathbb{R}}                      % Wes's shortcut command for the set of all real numbers
\newcommand{\C}{\mathbb{C}}                      % Wes's shortcut command for the set of all complex numbers
\newcommand{\Q}{\mathbb{Q}}
\newcommand{\N}{\mathbb{N}}
\usepackage{enumerate}
\renewcommand\thesection{\S\arabic{section}}
\newtheorem{theorem}{Theorem}
\newtheorem{lem}{Lemma}
\usepackage{mdframed}
\newcommand{\bslash}{\symbol{92}}
\newcommand{\upint}[1][2]{\overline{\int_{#1}^{#2}}}
\newcommand{\loint}[1][2]{\underline{\int_{#1}}^{#2}}
\usepackage{dirtytalk}
\newcommand{\bconditions}{\left\{\begin{aligned}}
\newcommand{\econditions}{\end{aligned}\right.}
\newcommand{\mat}{\begin{bmatrix}}
\newcommand{\trix}{\end{bmatrix}}
\newcommand{\dell}{\partial}
\begin{document}


\thispagestyle{empty}

\begin{center}
{\LARGE \bf Lecture 32: Line Integrals Concluded, Surface Integrals}\\
{\large Monday, 17 April 2023}\\
\end{center}

\section{Green's Theorem}

We now conclude our study of line integrals with Green's Theorem.

\begin{shaded}
\theorem (\textit{Green}) Let $F:\Omega\subseteq \R^2\to \R^2$ be of class $C^1$ on the open set $\Omega$. Let $\Gamma$ be a piecewise $C^1$ regular Jordan curve contained in $\Omega$ and let $D$ be the region consisting of $\Gamma$ and the enclosed by $\Gamma$. If $D\subseteq \Omega$ and $\Gamma$ is positively oriented, then 
$$
\oint_\Gamma F = \iint_D \frac{\dell F_2}{\dell x} -\frac{\dell F_1}{\dell y}.
$$
\end{shaded}

\begin{proof}
    For homework, we will prove the case where $D$ is a rectangle.
\end{proof}

\ex Let $F:\R^2\to\R^2$ be defined as
$$
F\left(\mat x \\ y \trix\right) = \mat 1 & 1 \\ -1 & 1 \trix \mat x \\ y \trix = \mat x + y \\ -x + y \trix
$$
and let $\Gamma$ be the unit circle, positively oriented. Evaluate
$$
\oint_\Gamma F.
$$

\textit{Solution:} We begin by parameterizing $\Gamma$ as $\gamma(t) = \mat\cos t \\ \sin t \trix$ for $0\leq t\leq 2\pi$. Then we have that
$$
\begin{aligned}
    \oint_\Gamma F &= \int_0^{2\pi} \langle F(\gamma(t)). \gamma'(t) \rangle ~dt \\
                &= \int_0^{2\pi} \left\langle \mat \cos t + \sin t \\ -\cos t + \sin t \trix , \mat -\sin t \\ \cos t \trix \right\rangle ~dt\\
                &=\int_0^{2\pi} -\sint^2 t - \cos^2 t~dt\\
                &=\int_0^{2\pi} -1~dt\\
                &=-2\pi.
\end{aligned}
$$
However, we also have that since $D=\overline{B(0,1)}$, we can evaluate the integral as
$$
\iint_D \frac{\dell F_2}{\dell x} - \frac{\dell F_1}{\dell y}= \iint_D -1-1 = -2(\mbox{area of } D) = -2\pi.
$$

\note Sometimes the expression 
$$
\frac{\dell F_2}{\dell x}-\frac{\dell F_1}{\dell y}
$$
is sometimes called the \textit{2-D curl} or \textit{rotation} of $F$, as it is a numerical indication of counterclockwise flow.

\section{Surface Integrals}
\subsection{Manifolds}
We will begin our discussion of surfaces with an informal definition of a surface and manifold.
\begin{mdframed}[backgroundcolor = blue!10]
\vspace{+0.1cm}
\defn An \textit{n-manifold} $\Sigma$ is an object for which each point on $\Sigma$ has a neighborhood (on the surface) that is homeomorphic to $\R^n$.
\end{mdframed}
\ex The unit circle in $\R^2$, that is, $S^1=\{(x,y)\in\R^2: x^2+y^2=1\}$ is a $1$-manifold.

Suppose $|x_0|<1$. We can pick an $\varepsilon>0$ such that $(x_0-\varepsilon,x_0+\varepsilon)\subseteq (-1,1)$. First, we project the interval onto the $x$-axis with the map $\pi$:
$$
\left(x_0,\sqrt{1-x_0^2}\right) \overset{\pi}{\mapsto} (x_0,0).
$$
This is a continuous bijection with a continuous inverse. Second, we translate this interval sideways with the map $\tau$: 
$$
(x_0-\varepsilon, x_0+\varepsilon)\overset{\tau}{\mapsto} (-\varepsilon,\varepsilon).
$$
The map $\tau$ is also a continuous bijection with a continuous inverse. Third and finally, let the bijection $f:(-\varepsilon,\varepsilon)\to \R$ be defined as 
$$
f(x)=\frac{x}{\varepsilon^2 -x^2}.
$$

Then we have that the composition 
$$
f\circ \tau \circ \pi
$$
is a homoemorphism, and thus $S^1$ is a $1$-manifold.

\subsection{The Cross Product in $\R^3$}
We will focus on 2-manifolds, surfaces in $\R^3$.
\ex If $T:\R^2\to\R^3$ is a linear transformation, then the image $T(\R^2)$ is a 2-manifold if and only if rank$(T)=2$.

Now, consider a surface $S$ in $\R^3$ parameterized by $r:A\subseteq \R^2\to \R^3$, where $A$ is connected an open. We need to extend the idea of a $C^1$ regular parameterization to surfaces. Then consider the grid curves of $r$. Let $(s_0,t_0)\in A$. Consider the curves $C_{s_0}(t) = r(s_0,t)$ and $C_{t_0}(s)=r(s,t_0)$ defined for $(s,t)$ near $(s_0,t_0)$. We have that $C_{s_0}'(t_0)$ is tangent to the its grid lines as is $C_{t_0}'(s_0)$ to its own. We would like $C_{s_0}'(t_0)$ and $C_{t_0}'(s_0)$ to be linearly independent, that way, locally near $r(s_0,t_0)$, we a get a well defined tangent plane
$$
r(s_0,t_0)+rC_{t_0}'(s_0)+tC_{s_0}'(t_0).
$$

In preparations for surface integrals, we ask the question: how can we estimate the area or the image of a small rectangle $[s_0,s_0+\Delta s]\times [t_0, t_0 + \Delta t]$ under a $C^1$ regular parameterization $r$ or a surface?

We first start with an aside on cross products.
\begin{mdframed}[backgroundcolor = blue!10]
\vspace{+0.1cm}
\defn Given two vectors $u,v\in\R^3$, the \textit{cross product} of $u$ and $v$is the vector
$$
u\times v = \mat u_2v_3 - u_3v_2\\  u_3v_1 - u_1v_3 \\ u_1v_2 - u_2v_1 \trix.
$$
\end{mdframed}
Some properties:
\begin{enumerate}
    \item The cross products of two vectors $u\times v$ is orthogonal to both $u$ and $V$ using the standard Euclidean norm.
    \item $u\times v = -(v\times u)$.
    \item $\|u\times v\| = \|u\| \|v\|\sin\theta$, where $\theta\in [0,\pi]$ is the angle between $u$ and $v$. This is also the same thing as the area of the parallelogram formed by $u$ and $v$.
\end{enumerate}

\subsection{Surface Integrals}
We now answer the question we asked earlier. Instead of writing $C_{s_0}' (t)$ and $C_{t_0}'(s)$, we can write
$$
r_s(s,t) = \mat \dell r_1 / \dell s \\ \dell r_2 / \dell s \\ \dell r_3 / \dell s\trix ,\quad 
r_t(s,t) = \mat \dell r_1 / \dell t \\ \dell r_2 / \dell t \\ \dell r_3 / \dell t\trix ,
$$
where all partials are evaluated at $(s,t)$. Notice that these partials are just the columns of the Jacobin for $Dr(s,t)$. Now consider the linear approximation of the function near $(s_0,t_0)$:
\begin{align*}
    r(s_0+\Delta s ,t_0)&\approx r(s_0,t_0) + r_s(s_0,t_0)\Delta s\\
    r(s_0 ,t_0+\Delta t)&\approx r(s_0,t_0) + r_t(s_0,t_0)\Delta t.
\end{align*}
Then the are of the image of a rectangle under $r$ is approximately
$$
\|r_s(s_0,t_0)\Delta s \cross r_t(s_0,t_0)\Delta t \| = \|r_s\times r_t\|\Delta s \Delta t.
$$

\begin{mdframed}[backgroundcolor = blue!10]
\vspace{+0.1cm}
\defn Assume $\Omega\subseteq \R^2$ is open, connected, and bounded. Let $S$ be the image of $\Omega$ under a $C^1$ regular parameterization $r:\Omega\to \R^3$. Assume $S$ is bounded. Now, let $f$ be continuous, bounded, and real valued on $\overline{ r(\Omega)}$. Then the we define the \textit{surface integral} $\int_S f$ as
$$
\iint_\Omega f(r(s,t)) \|r_s\times r_t\|.
$$
In particular, $\int_S 1$ is a surface area.
\end{mdframed}
\end{document}