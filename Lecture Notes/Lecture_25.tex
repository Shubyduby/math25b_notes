\documentclass[11pt]{article}
\usepackage[utf8]{inputenc}	% Para caracteres en español
\usepackage{amsmath,amsthm,amsfonts,amssymb,amscd}
\usepackage{multirow,booktabs}
\usepackage[table]{xcolor}
\usepackage{fullpage}
\usepackage{lastpage}
\usepackage{enumitem}
\usepackage{fancyhdr}
\usepackage{mathrsfs}
\usepackage{wrapfig}
\usepackage{setspace}
\usepackage{calc}
\usepackage{stmaryrd}
\usepackage{multicol}
\usepackage{cancel}
\usepackage[retainorgcmds]{IEEEtrantools}
\usepackage[margin=3cm]{geometry}
\usepackage{amsmath}
\newlength{\tabcont}
\setlength{\parindent}{0.0in}
\setlength{\parskip}{0.05in}
\usepackage{empheq}
\usepackage{framed}
\usepackage[most]{tcolorbox}
\usepackage{xcolor}
\colorlet{shadecolor}{orange!15}
\parindent 0in
\parskip 12pt
\geometry{margin=1in, headsep=0.25in}
\theoremstyle{definition}
\newtheorem{defn}{Definition}
\newtheorem{reg}{Rule}
\newtheorem{exer}{Exercise}
\newtheorem{note}{Note}
\newtheorem{prop}{Proposition}
\newtheorem{ex}{Example}
\newcommand{\R}{\mathbb{R}}                      % Wes's shortcut command for the set of all real numbers
\newcommand{\C}{\mathbb{C}}                      % Wes's shortcut command for the set of all complex numbers
\newcommand{\Q}{\mathbb{Q}}
\newcommand{\N}{\mathbb{N}}
\usepackage{enumerate}
\renewcommand\thesection{\S\arabic{section}}
\newtheorem{theorem}{Theorem}
\newtheorem{lem}{Lemma}
\usepackage{mdframed}
\newcommand{\bslash}{\symbol{92}}
\newcommand{\upint}[1][2]{\overline{\int_{#1}^{#2}}}
\newcommand{\loint}[1][2]{\underline{\int_{#1}}^{#2}}
\usepackage{dirtytalk}
\newcommand{\bconditions}{\left\{\begin{aligned}}
\newcommand{\econditions}{\end{aligned}\right.}
\newcommand{\mat}{\begin{bmatrix}}
\newcommand{\trix}{\end{bmatrix}}
\newcommand{\dell}{\partial}
\begin{document}


\thispagestyle{empty}

\begin{center}
{\LARGE \bf Lecture 25: Maxima, Minima, Optimization}\\
{\large Wednesday, 29 March 2023}\\
\end{center}

Recall these facts about optimization:
\begin{enumerate}
    \item A real-valued function $f$ has a local maximum at a point $x_0$ in its domain if there exists an open neighborhood $\Omega$ of $x_0$ such that $x\in\Omega$ is in the domain of $f$, then $f(x)\leq f(x_0)$.
    \item If $f:(a,b)\to\R$ is differentiable at $f$ has a local extremum at $c\in(a,b)$, then $f'(c)=0$. 
    \item The extreme value theorem: if $f:[a,b]\to\R$ is continuous, then $f$ achieves an absolute maximum and minimum somewhere on $[a,b]$. We search for these at critical points or endpoints.
\end{enumerate}
But what do we mean by critical points?

\begin{mdframed}[backgroundcolor = blue!10]
\vspace{+0.1cm}

\defn Let $f:\Omega\subseteq \R^n\to \R$, where $\Omega$ is open. A \textit{critical point} of $f$ is a point $x_0\in\Omega$ such that either $\nabla f(x_0)=0$ or $\nabla f(x_0)$ does not exist.

\end{mdframed}

\prop Let $f:\Omega\subseteq \R^n\to\R$ with $\Omega$ open be differentiable and assume $f$ has a local extremum some $x_0\in \Omega$. Then $\nabla f(x_0)=0$.

\begin{proof}
    Let $f$ achieve a local maximum at $x_0$, and assume for the sake of contradiction that $\nabla f(x_0)\neq 0$. Now let $\varepsilon=\|\nabla f(x_0)\|/2>0$. Let $\delta>0$ such that $B(x_0,\delta)\subseteq \Omega$, which we can find since $\Omega$ is open, and $\|x-x_0\|<\delta$ implies that $\|f(x)-f(x_0)-\nabla f(x_0)(x-x_0)\|\leq \varepsilon\|x-x_0\|$, which exists as well since $f$ is differentiable. In particular, let the unit vector $u=\nabla f(x_0)/\|\nabla f(x_0)\|$ and consider $x=x_0+hu$ with $|h|<\delta$. Since $|h|<\delta$, we have that
    $$
    |f(x_0+hu)-f(x_0)-\nabla f(x_0)\cdot (hu)|\leq \varepsilon|h|
    $$
    So 
    $$
    |f(x_0+hu)-f(x_0)-\|\nabla f(x_0)\| h|\leq \varepsilon|h|.
    $$
    In particular, 
    $$
    -\varepsilon|h|\leq f(x_0+hu)-f(x_0)-\|\nabla f(x_0)\|h.
    $$
    If $0<h<\delta$,
    $$
    -\varepsilon h\leq f(x_0+hu)-f(x_0)-\|\nabla f(x_0)\|h,
    $$
    so by how $\varepsilon$ is defined we have that
    $$
    f(x_0+hu)\geq f(x_0)+\frac{1}{2}\|\nabla f(x_0)\|h,
    $$
    so $f(x_0)$ is not a local maximum, a contradiction \mbox{\lightning}.
    
\end{proof}

\section{Extreme Value Theorem}

If $f:K\subseteq \R^n\to \R$, where $K$ is compact, is continuous, the extreme value theorem applies. Absolute extrema then are achieved on $\dell K$ or on a critical point of $\mathrm{int}(K)$. 

\ex Consider the function $f:\R^2\to \R$ defined as $z\mapsto \|z\|$. Restrict the domain to the ellipse
$$
K=\left\{(x,y)\in \R^2:\frac{x^2}{4}+\frac{y^2}{9}\leq 1\right\}.
$$

Note that the function $f$ is continuous because for any $z_1,z_2\in K$, we have
$$
|f(z_1)-f(z_2)|=|\|z_1\|-\|z_2\||\leq \|z_1-z_2\|.
$$
Let us find the critical points. We have
$$
f(x,y)=\sqrt{x^2+y^2}\implies \frac{\dell f}{\dell x}=\frac{2x}{2\sqrt{x^2+y^2}}.
$$
When we do the same the the partial derivative with respect to $y$, we find that the only possible critical point is at $(0,0)$. At the origin,
$$
\lim_{h\to 0} \frac{f(h,0)-f(0,0)}{h}=\lim_{h\to 0} \frac{|h|}{h},
$$
so $\nabla f(0,0)$ does not exist and thus $f$ is not differentiable at $(0,0)$. Then $(0,0)$ is the only critical point. We know $f(0,0)=0$.

Now we scan the boundary. We can begin by parameterizing the boundary:
$$
\gamma(t)=\mat 2\cos t \\ 3\sin t \trix.
$$
Then we optimize $f(\gamma(t))$ with our familiar single variable optimization.

\note Critical points are not always maxima of minima! You will often run into critical points that do now yield optimized points.

\ex The function $f:\R\to\R$ defined as $f(x)=x^3$ has a critical point at $x=0$ but that is evidently not an extremum.

\ex Consider the function $f:\R^2\to\R$ defined as $f(x,y)=x^2-y^2$ (visualize this funciton, it looks something like a saddle). The gradient is $\nabla f(x,y)=\mat 2x &-2y\trix$ so we have a critical point at $(0,0)$. But it is not an extremum, it is a \textit{saddle point} since every open neighborhood of $(0,0)$ contains points $(x_1,y_1)$ with $f(x_1,y_1)>f(0,0)$ and $(x_2,y_2)$ wit $f(x_2,y_2)<(0,0)$. 

\section{Second Derivative Test}
To do optimization, we want to generalize the second derivative test. Recall that the second derivative test from $\R\to\R$ was that if $f\in C^2(a,b)$, $f:(a,b)\to \R$, $f'(x_0)=0$, and $f''(x_0)>0$, then $f$ as a local minimum at $x_0$. This requires the use of bilinear forms.

\begin{mdframed}[backgroundcolor = blue!10]
\vspace{+0.1cm}

\defn A bilinear form $B:\R^n\times\R^n\to \R$ is called \textit{positive definite} if $B(x,x)\geq 0$ for all $x\in \R^n$ with equality if and only if $x=0$.

\defn A bilinear form is \textit{positive semidefinite} if $B(x,x)\geq 0$ for all $x\in\R^n$

\end{mdframed}

\note Since $B(y,z)=y^T Mz$ where $M_{ij}=B(e_i,e_j)$, we will say that a matrix $M$ is \textit{positive definite} if $x^TMx >0$ for all $x\in\R^n\bslash\{0\}$.

\prop (\textit{Second Derivative Test}) Let $f:\Omega\subseteq \R^n\to\R$ where $\Omega$ is open be of class $C^2$. Suppose $f$ has a critical point at $x_0\in\Omega$. If the bilinear form $D^2f(x_0)$ is positive definite, then $f$ has a local minimum at $x_0$. 

\begin{proof}
    Our goal here is to show that there exists some $\delta>0$ such that $B(x_0,\delta)\subseteq \Omega$ and if $\|x-x_0\|<\delta$ then $f(x)\geq f(x_0)$. Like the earlier proof today, we will single out a special direction in $\R^n$. Since $D^2f(x_0)(x,x)>0$ for all $x\in\R^n\bslash\{0\}$, this inequality is true for unit vectors $u\in\R^n$. Let 
    $$
    S^{n-1}=\{u\in\R^n:\|u\|=1\}.
    $$
    This sphere is compact and $D^2f$ is continuous since $f$ is of class $C^2$, so the map defined as $u\mapsto D^2f(x_0)(u,u)$ is continuous and real valued. By the extreme value theorem, there exists a $u^*\in S^{n-1}$ such that $\varepsilon=D^2f(x_0)(u^*,u^*)$ is minimal. Note that $\varepsilon>0$. So we look at the value of $D^2f(x_0)(x,x)$ for $x\neq 0$. We know $x=x/\|x\|\cdot\|x\|$, so
    \begin{equation}
    D^2f(x_0)(x,x)=\|x\|^2 D^2f(x_0)\underbrace{\left(\frac{x}{\|x\|},\frac{x}{\|x\|}\right)}_{\mbox{unit vectors}}\geq \varepsilon\|x\|^2.
    \end{equation}
    We are almost ready to apply Taylor-Lagrange! Since $\Omega$ is open and $D^2f$ is continuous, we can choose a $\delta>0$ such that $\|x-x_0\|<\delta$ implies 
    \begin{equation}
    \|D^2f(x)-D^2f(x_0)\|<\varepsilon/2
    \end{equation}
    and $B(x_0,\delta)\subseteq\Omega$. The convexity of the ball allows us to apply Taylor-Lagrange:
    $$
    f(x)=f(x_0)+\underbrace{\nabla f(x_0)(x-x_0)}_{=0}+\frac{1}{2}Df^2(c)(x-x_0,x-x_0)
    $$
    for some $c$ on the line segment connection $x$ to $x_0$. By inequalities (1) and (2), we have that
    \begin{align*}
        D^2(c)(x-x_0,x-x_0) & = D^2f(c)(x-x_0,x-x_0)\\
        &\quad-D^2f(x_0)(x-x_0,x-x_0)+D^2f(x_0)(x-x_0,x-x_0)\\
        &= D^2f(x_0)(x-x_0,x-x_0)\\
        &\quad - (D^2(x_0)(x-x_0,x-x_0) - D^2(c)(x-x_0,x-x_0))\\
        &\geq D^2f(x_0)(x-x_0,x-x_0)\\
        &\quad - \|D^2f(x_0)(x-x_0,x-x_0)-D^2(c)(x-x_0,x-x_0)\|\\
        &\geq \varepsilon\|x-x_0\|^2-\frac{\varepsilon}{2}\|x-x_0\|^2\\
        &=\frac{\varepsilon}{2}\|x-x_0\|^2.
    \end{align*}
    Thus,
    $$
    f(x)\geq f(x_0)+\frac{\varepsilon}{4}\|x-x_0\|^2,
    $$
    so $f$ has a strict local minimum at $x_0$.
\end{proof}
\end{document}